We've made foundational progress in two areas: symmetric lenses and edit
lenses.

\section{Future Work}
\label{sec:future}

\begin{enumerate}
    \item hyperlenses
    \item more syntax
        \begin{itemize}
            \item edit modules for recursive types, fold/unfold lens combinators,
                investigate shape alignment
            \item edit modules for more container shapes, esp. graphs (for
                model-driven development) and relations (for databases)
        \end{itemize}
    \item algebraic properties
        \begin{itemize}
            \item associative sum
            \item symmetric monoidal structure of tensor product edit lens
            \item monoid laws for the various edit monoids
            \item partially-ordered homsets: perhaps we can't get a product or sum
                in our category, but could get a variant where the required
                equations are replaced by inequalities
        \end{itemize}
    \item wiring diagrams; some text copied here from the original
        complements paper:

        More speculatively, it is a well-known folklore result that
        symmetric monoidal categories are in 1-1 correspondence with wiring
        diagrams and with first-order linear lambda calculus. We would like
        to exploit this correspondence to design a lambda-calculus-like
        syntax for symmetric lenses and perhaps also a diagrammatic
        language. The linear lambda calculus has judgments of the form
        $x_1{:}A_1,\dots,x_n{:}A_n\vdash t:A_0$, where $A_0,\dots,A_n$ are
        sets or possibly syntactic type expressions and where $t$ is a
        linear term made up from basic lenses, lens combinators, and the
        variables $x_1,\dots,x_n$. This could be taken as denoting a
        symmetric lens $A_1\otimes\dots\otimes A_n\lens A_0$. For example,
        here is such a term for the lens $\mathit{concat}'$ from
        \S\ref{concatprime}:
        \[\begin{array}{@{}l}
        z{:}\Unit \oplus A \otimes A\LIST\otimes A\LIST\vdash
            \begin{array}[t]{@{}l}
            \textit{match}\ z\ \textit{with} \\
            \quad \mid\mlinl\unit \mapsto \const_{\NIL}\op\\
            \quad\mid\mlinr(a,al,ar)\mapsto \mathit{concat}(a\CONS al,ar)
            \end{array}
        \end{array}
        \]
        The interpretation of such a term in the category of lenses then
        takes care of the appropriate insertion of bijective lenses for
        regrouping and swapping tensor products.
    \item typed edits/complements; some text moved here from the related
        work section:

        One can view the two approaches as two extremes, with on one end
        graphs with a single node representing all possible repository
        states and on the other end graphs with many nodes where each node
        represents a single repository state. There may be a middle ground
        in which graph nodes each represent many possible repository states;
        the hope then would be that one could keep the benefit of a total
        edit application function while reusing single edits on many
        different states. For example, for list edits, one might consider a
        graph with one node for each possible length of list. Then one would
        have, for example, deletion edges $\mldelete : m \dedge n$ when
        $m<n$; such an edge must store marginally more information than our
        edit module did (the domain and codomain length rather than a single
        number telling their difference), but the set of repositories to
        which it applies is much more clearly delimited. Attempting to
        recast the edit modules and lenses proposed above in this light
        would be an interesting area for future work.
    \item parsing/unparsing string edits to structured edits
    \item automatically discover weight functions for symmetric lens
        iteration
    \item laws about undo-able edits, and requiring that \emph{edits} ``roundtrip''
    \item how closely do symmetric lens and edit lens constructions like
        tensor product and sum correspond? (can we get a theorem like the
        one saying how asymmetric lens constructions lift to symmetric lens
        constructions?)
    \item how closely do (symmetric) edit lenses and asymmetric delta lenses
        correspond? (can we get a theorem like the one saying how to
        factor a symmetric lens into two asymmetric ones?)
    \item implementation

\section{Closing Thoughts}
Nevertheless lenses are part of an ecosystem of bidirectional techniques
aimed at the common task of maintaining replicated data and we expanded
what's known about incrementalizing and symmetrizing them, an important step
in the development of such techniques.
\end{enumerate}
