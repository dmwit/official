Our work has identified several areas for improvement in the foundations of
existing bidirectional transformation tools. The development of symmetric
lenses enables both repositories associated with a lens to store locally
interesting information. Considering stateful, rather than pure,
transformations enables the lens to store this local information on the
side; though a theory of behavioral equivalence is needed to restore
equational reasoning in the presence of this state. This is an improvement
on the asymmetric lens framework's need for a canonical, centralized
repository, but the behavior of many of the actual lenses proposed in
Chapter~\ref{chap:symmetric} remain undesirable in several ways. Most
significantly, they employ a somewhat na\"ive strategy for divining the
connection between original and updated repositories. This manifests itself
as lenses which follow the letter of the law---produce synchronized
repositories---but mangle the meaning of the repositories during
synchronization by inappropriately mixing and matching data.

Edit lenses address this problem by elevating the status of the data
describing how original and updated repositories are connected. Treating
edits as first-class means that the lenses need not guess about alignment
information, and consequently will mangle the meaning of repositories less
often. Indeed, as a side effect, promoting edits in this way allows the
creation of stronger laws, so that the letter of the law and the spirit of
the law are no longer quite so far from each other. We have also shown that
one may realize performance benefits with this approach with careful design
of the edit languages being manipulated by edit lenses.

Related approaches to this problem are summarized in
Table~\ref{tab:related-commentary}; the edit lens framework is the first
approach to address all of the issues raised above. Asymmetric delta lenses
seem like a promising approach for their ability to handle alignment well,
but the formalism does not accomodate small edits well even if syntax could
be designed that could operate on them. Symmetric delta lenses extend these
to the symmetric setting, but the reasoning principles that require
behavioral equivalence discussed in conjunction with our symmetric lenses
are not considered, and the framework itself is not yet instantiated with a
syntax. The algebraic study of asymmetric lenses exposes many surprising
features of edits, but does not address many of the issues needed to create
a practical system. Matching lenses and annotation-based lenses are very
natural extensions of asymmetric, state-based lenses, but are also
conservative in many ways. The more radical changes proposed in edit lenses
allow for symmetric operation and eliminate the need to pass repositories to
the lenses. Notably, annotation-based lenses extend a variant of the
asymmetric, state-based lenses that allows for the construction of a lens
that duplicates information, making it the only approach that takes a
serious stab at alignment handling while enabling this lens. Finally,
constraint maintainers tackle many practical issues, but do not fully
explore the power of edits and lack the ability to perform sequential
composition, a key piece of syntax.

\begin{table}
    \begin{center}
        \begin{tabularx}{\linewidth}{m{4.4em}|XXXX}
        & Alignment & Symmetry & Performance & Syntax
        \\\hline
%        TGGs            &\N&\Y&\N&\Y&\\
        symm. state     &\N[very bad]
                        &\Y[yes; requires equivalence]
                        &\N[no]
                        &\Y[mostly domain agnostic]
        \\\hline
        asymm. $\delta$ &\Y[explicit alignments]
                        &\N[not a goal]
                        &\N[edits include repositories]
                        &\Y[via alternate framework]
        \\\hline
        symm. $\delta$  &\Y[edits]
                        &\Y[yes, but equiv. not explored]
                        &\N[edits include repositories]
                        &\N[alternate frameworks not instantiated]
        \\\hline
        algebraic       &\Y[edits]
                        &\N[no]
                        &\N[possibly, but unexplored]
                        &\N[not a goal]
        \\\hline
        matching        &\Y[mapping from holes to holes]
                        &\N[no]
                        &\N[repository and alignment information both
                        processed]
                        &\Y[variants of most AS-lens combinators]
        \\\hline
        annotated       &\Y[insertion, deletion, modification markers]
                        &\N[no]
                        &\N[alignment information includes repository]
                        &\Y[includes $\diag \in X \lens X \times X$]
        \\\hline
        const. maint.   &\Y[uninterpreted edits]
                        &\Y[yes; does not require equiv.]
                        &\N[no; all edits relative to $\init$]
                        &\Y[many primitives, but no composition]
    \end{tabularx}
    \end{center}
    \caption{Feature coverage for various alternatives to edit lenses}
    \label{tab:related-commentary}
\end{table}

Though the issues that edit lenses handle are important ones, there remain
many other foundational issues that one might want to tackle to strengthen
the practicality of bidirectional transformation frameworks. The following
section surveys a few of the most pressing needs.

\section{Future Work}
\label{sec:future}

\paragraph*{Hyperlenses}
The lens framework focuses itself on the problem of synchronizing two
repositories at a time. Consequently, current lenses do not generalize
smoothly to more than two pieces of data, but many real-world scenarios
involve synchronizing many (potentially quite small and loosely related)
repositories. One example (which we are not the first to
propose~\cite{macedotowards}) would be a multi-directional spreadsheet,
where we treat each cell as a repository. Some cells are computed from
others; these computations are the transformations that one might like to
bidirectionalize. There seem to be a variety of additional challenges
associated with generalizing from bi-directional to many-directional
updating, chief among them being a significantly larger update space to
search through on each synchronization action. We have explored a few
restricted settings---for example, where no repository is connected to
another in two different ways, or where all repositories are numbers and all
connections are linear functions---that seem to admit partial solutions, but
none are really satisfactory. There seem to be deep connections to the
literature on constraint propagation and (in the special case of
spreadsheets) computer algebra systems.
% TODO: add a couple citations for constraint propagation and CAS

% TODO: possible citation: Survey: Practical Applications of Constraint
% Programming, Mark Wallace
% see also the bibliography in
% https://github.com/dmwit/triple-threat/blob/master/spreadsheet_resources.txt

\paragraph*{Additional Syntax}
We have noted in our discussions above several places where one could wish
for a more expressive collection of lens constructions. In particular, we
were not able to recapitulate the symmetric lens' development of fold and
unfold lens combinators for recursive types, in part because it is not clear
how to build edit modules for recursive types in a compositional way. (Edits
to ``roll'' and ``unroll'' one layer of the recursion are not enough: as
with lists, one wants a way to shuffle data between depths and across pairs
and sums; this is the part that seems tricky.) Even restricting our view to
containers, it would be interesting to investigate edit modules for more
container shapes, especially graphs---the basis of the data model usually
used in model-driven development---and relations---the data model usually
used in databases. More speculatively, it is well-known that symmetric
monoidal categories are closely connected to wiring
diagrams~\cite{selinger2011survey} and to first-order linear lambda
calculus~\cite{seely1987linear}. Perhaps one could exploit this
correspondence to design a lambda-calculus-like syntax or diagrammatic
language for symmetric or edit lenses. The linear lambda calculus has
judgments of the form $x_1{:}A_1,\dots,x_n{:}A_n\vdash t:A_0$, where
$A_0,\dots,A_n$ are sets or possibly syntactic type expressions and where
$t$ is a linear term made up from basic lenses, lens combinators, and the
variables $x_1,\dots,x_n$. This could be taken as denoting a symmetric lens
$A_1\otimes\dots\otimes A_n\lens A_0$. For example, here is such a term for
the lens $\mathit{concat}'$ from \S\ref{concatprime}:
\[\begin{array}{@{}l}
z{:}\Unit \oplus A \otimes A\LIST\otimes A\LIST\vdash
    \begin{array}[t]{@{}l}
    \textit{match}\ z\ \textit{with} \\
    \quad \mid\mlinl\unit \mapsto \const_{\NIL}\op\\
    \quad\mid\mlinr(a,al,ar)\mapsto \mathit{concat}(a\CONS al,ar)
    \end{array}
\end{array}
\]
The interpretation of such a term in the category of lenses then takes care
of the appropriate insertion of bijective lenses for regrouping and swapping
tensor products.
% TODO: fit the lack of a conditional edit lens in somewhere?

\paragraph*{Algebraic Properties}
A number of algebraic oddities have cropped up during our development which
it would be nice, as a matter of polish, to settle one way or another. The
status of sums (and in particular injection lenses) has been somewhat in
question for some time, so it is nice that in the symmetric lens case we
have settled this question by elucidating the symmetric monoidal structure
available. On the other hand, this makes the lack of associativity for
tensor sum in the edit lens category all the more surprising; it may be
interesting to pursue (or prove impossible) an associative tensor sum.
Similarly, we have shown that our tensor product has many of the properties
we expect of a symmetric monoidal category structure, but not quite all. We
conjecture that adding appropriate monoid laws would resolve this problem;
and anyway adding appropriate monoid laws in all the modules discussed is a
serious undertaking of its own worth consideration. Finally, it has been
suggested by a reviewer of one of our papers that, although we cannot have a
categorical product or sum in the category of lenses, it may be worth
considering placing a partial order on lenses. Perhaps this would enable a
variant of the categorical product where the usual defining equations of a
product are replaced by inequalities.

\paragraph*{Implementation}
Effort spent on an implementation could be aimed at a variety of purposes.
One might be interested in verifying whether the performance promises of
edit lenses could be realized by running experiments. For example, one might
imagine comparing the size of typical repository edits to the size of the
repositories as the repository grows; comparing the runtime of edit
translation to the runtime of a more traditionally designed lens; or
analyzing any of half a dozen other metrics. Alternately, one could focus on
breadth rather than depth, implementing a variety of transformations, to
find out whether the syntax developed here is expressive enough. There are
also many practical tools that may benefit from edit lenses: file
synchronizers keeping two file systems synchronized, text editors keeping
parse trees synchronized, database backends keeping queries synchronized
with data, log summarizers keeping summaries and log files synchronized,
software model transformations keeping architectural diagrams and code
synchronized, perhaps even mobile phone applications keeping a client's
display and server data synchronized may all benefit from bidirectional
techniques.

\paragraph*{Miscellaneous Extensions}
Besides the broad categories discussed above, there are a handful of other
curiosities suggested throughout the development which we gather here.
During the development of iterator symmetric lenses, it was observed that
correctness of the lens depends on the existence of an appropriate weight
function guaranteeing termination. We anticipate that this function will be
simple in the majority of cases; automatically discovering it for a broad
class of lenses seems plausible and would remove a significant annoyance
from the lens programmer. Next, in the passage from asymmetric, state-based
lenses to symmetric lenses, we gave a theorem connecting various asymmetric
lens constructions to symmetric lens constructions. It would be nice to know
whether this connection could be extended to edit lenses. For example, do
edit lens constructions like tensor product correspond in some way to the
lifting of symmetric lens tensor product? One similarly wonders whether
there is an edit lens analog of the theorem showing how to split a symmetric
lens into two asymmetric lenses. On a slightly different line of inquiry,
one may wonder just how canonical our choices of edit structure and lens
laws are. Several related lines of work have proposed more intricate
structures---for example, a popular choice is to require that edits have
some way of being undone, either with an inverse or something weaker---and
stronger laws---for example, requiring that $\dputr$ and $\dputl$ produce
minimal edits by adding a triple-trip law---without instantiating their
frameworks. Perhaps enforcing stronger requirements would suggest ways to
improve the behavior of our existing constructions, and on the other hand
perhaps our constructions would reveal that some of the requirements are too
strong.

Another idea suggested by related work is to consider typed edits. In our
development of edit lenses, we allow partial edits so that we may represent
edits that work on some, but not all, repositories in a uniform way. We then
go to great lengths to assure that the partiality is purely formal: we have
a theorem showing that lenses never introduce partiality where none was
before. On the other hand, the symmetric delta lens approach has no such
problem---each edit is applicable to one and only one repository---but pays
the price of having to duplicate the modifications which can apply to many
repositories~\cite{Diskin-Delta11}. Perhaps it would be possible to find a
middle ground by designing a typed edit language, in which edits may apply
to many repositories (that share a type), but where the types are specific
enough that all edits are total. For example, for list edits, one might
consider having one type for each possible length of list. Then one would
have, for example, deletion edges $\mldelete : m \dedge n$ when $m<n$; such
an edge must store marginally more information than our edit module did (the
domain and codomain length rather than a single number telling their
difference), but the set of repositories to which it applies is much more
clearly delimited.

Finally, many asymmetric lens frameworks ground their constructions by
showing how to use them with a string-based data model. This significant
task was not undertaken here\footnote{The use of ``significant'' in this
sentence is true in both the ``important'' and ``difficult'' senses.}.
Preliminary efforts suggest that this is an especially difficult task in the
world of edit lenses: though we have several decades of experience parsing
strings into more structured data, there is comparatively little effort
expended on parsing string edits into edits on more structured data.

\section{Closing Thought}
\label{sec:closing}
Though there are many opportunities for further improvements, edit lenses
are part of a growing ecosystem of bidirectional techniques. In this
arena, our development expands what is known about incrementalizing and
symmetrizing bidirectional transformations---an important step in the
development of practical tools for the common task of maintaining replicated
data.
