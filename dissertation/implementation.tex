\section{Introduction}
\label{sec:impl-intro}
Having developed the theory of lenses and instantiating the framework with a
syntax, we now give an exposition on preliminary efforts to instantiate the
syntax as a concrete program. Our work on a prototype has two main purposes.
% TODO: the wording of this first purpose needs some serious polish
The edit lens framework is predicated on a relatively abstract, algebraic
data model, whereas long-term data storage on computers typically employs a
fairly low-level model based on strings. When only the data is important,
these two realms are typically connected by defining a parser that processes
strings and produces a more structured representation, as well as a
formatter that produces a string representation of a given structure. For
edit lenses, however, not only the data is important; one also wants access
to the edits made to the data. So the primary goal is to investigate what
extensions are needed to describe the connection between edits to strings
and edits to structured data. A secondary goal is to validate that the
fundamental edit lens design is complete; producing a few example
transformations gives an opportunity for any unforeseen infelicities to rear
their head. In the pursuit of these goals, we discuss two artifacts: first,
a core library which closely models the edit lens theory given in
Chapter~\ref{chap:edits}; and second, a demonstration program that
synchronizes two simple, text-based databases according to a predetermined
lens. The latter task involves building a text-editing GUI, connecting the
lens to the GUI, and validating and extracting edits from user actions,
tasks that fall outside the realm of the existing edit lens theory.

We have chosen to implement our demo in Haskell, a language which encourages
high abstraction levels, supports rapid prototyping, and has good library
support. Because one of our primary goals was experimentation, we wanted to
retain a lightweight approach throughout; in particular we chose not to
begin with a mechanization of the theory in a dependently-typed language.
(In retrospect, while it still seems worthwhile to have avoided reproving
all of our results in Coq from an experimentation point of view, it is not
clear that avoiding dependent types entirely was beneficial. We will see in
Section~\ref{sec:impl-details} that the way we used Haskell's typeclass
mechanism would really have benefited from dependent types; for example, we
will wish for the ability to define new types for each choice of $\init$
value.) We also investigated extending Boomerang~\cite{Boomerang07}, an
existing asymmetric, state-based string lens implementation. Boomerang is
very complete, and consequently would have required many tangential coding
efforts; to avoid distractions, we chose to take a less feature-complete
route. However, we retained Boomerang's choice of string-based data model
since, as discussed above, this closely matches real-world scenarios.

Our primary challenge, which we will discuss in detail below, can be broadly
described as parsing. With edit lenses, there are always two domains of
discourse: the collection of repositories and the collection of edits.
Repositories store ordinary data, and the problem of connecting strings with
structured data is well-studied under the umbrella of parsing. (Turning
structured data into a string---often called serialization---is typically a
significantly simpler task.) However, standard parsing techniques---even
incremental techniques purportedly designed for making it easy to maintain a
correct parse tree in the presence of ongoing updates---do not adequately
describe the connection between string modifications and edits in the sense
described in Chapter~\ref{chap:delta}. We have proposed a few heuristics
that seem to behave acceptably in a number of standard cases; however, they
are relatively special-purpose (tailored to the file format under
consideration here) and do not adequately reflect all user actions as
analogous edits. This seems like a promising area for future efforts.

\section{Usage Example and Functionality}
\label{sec:impl-usage}
\begin{enumerate}
    \item describe database format + lens connecting them
    \item screenshots showing synchronized databases
    \item a few simple edits that get reflected; and why we can't reflect before the edit is completed
\end{enumerate}

% TODO: this reverse highlighting experiment turned out a little strange
% because of the document's white background; let's try more orthodox
% highlighting or a circle around the change or something like that instead
\begin{figure}
    \centering
    \protofig1{An initial pair of databases in two text editing panes.}
    \hfil
    \protofig2{Insertion on the right introduces some default data on the
    left.}
    \vspace{4ex}

    \protofig3{Deleting a row from either side is reflected to the other
    automatically.}
    \hfil
    \protofig4{A default country is used for the new row on the right.}
    \vspace{4ex}

    \protofig5{Correcting the country has no effect\ldots}
    \hfil
    \protofig6{\ldots but correcting the spelling of either name corrects
    both.}
    \vspace{4ex}

    \protofig7{More bizarre edits, like this deletion that spans
    records\ldots}
    \hfil
    \protofig8{\ldots reset the alignment only for the affected region.}
    \vspace{4ex}

    \caption{A demonstration use of the prototype, using the composers lens}
    \label{fig:prototype-screenshots}
\end{figure}

\section{Implementation Details}
\label{sec:impl-details}
annotated code, may describe selected functions or maybe all functions (?)
\begin{enumerate}
    \item edits + edit application: the Module type class
    \item lenses, simple lenses, lens+module triples
    \item basic lenses + maybe a few combinators
    \item unparsing (needs more motivation and explanation, or less if you consider this as not being reusable code)
    \item parsing (needs more motivation and explanation, or less if you consider this as not being reusable code)
    \item connecting to a GUI + storing complements in ref cells
    \item a `bad' choice - modules are based on type classes instead of being records
\end{enumerate}

\section{Conclusion}
\label{sec:impl-conclusion}
\begin{enumerate}
    \item Message: this is an existing library and an associated GUI that extracts alignment information from user actions
    \item Message: the existence of a working library is an indication that nothing important was overlooked in the theoretical foundation
    \item Message: this library could be used for further studies of edit lenses beyond the scope of the current work
        \begin{enumerate}
            \item original purpose: convert a string edit into a tree edit (resulted in a hard problem)
            \item demonstrate usability of the syntax by generating some practical examples
            \item show a performance advantage
            \item demonstrate a practical application of lenses (we have a long list of ideas about this)
        \end{enumerate}
    \item Outcome: we need new techniques for some parts, but core library can be elegant
\end{enumerate}

\section{Full code}
\label{sec:impl-code}
perhaps appendix, or pointers to hosting, or an attachment to the dissertation, or some such thing
