\begin{itemize}
    \item Bidirectional programming is advantageous because it is concise
        (one iteration only) and has proven synchronization laws
\end{itemize}
% TODO: introduce jargon like ``repository'' and friends in here

\section{Asymmetric Lenses}
\label{sec:asymmetric-lenses}
\begin{itemize}
    \item Asymmetric Lenses
        \begin{itemize}
            \item Define what an asymmetric lens is and how its used
            \item This was the initial approach to lenses
            \item Asymmetric lenses needed to mature to add functionality
        \end{itemize}
    \item The extra functionality should address practical issues arising
        from the synchronization of the repositories
        \begin{itemize}
            \item alignment: there may be many ways to produce a
                synchronized pair, and we must give tools for disambiguating
                \begin{itemize}
                    \item to do a good job, a lens needs information about
                        what happened that's hard to recover locally
                    \item structural matching clearly does a bad job -- and
                        the naive asymmetric model can only make local
                        decisions, none of which are going to be good enough
                    \item producing descriptions of the global information
                        itself is messy, and you want that factored out --
                        you want a theory of how global information is
                        processed without worrying about it's produced
                \end{itemize}
            \item symmetry: there may be no canonical repository that
                contains all the data
            \item size: sometimes repositories are too large to facilitate
                functional computations or to transmit
        \end{itemize}
\end{itemize}

One well-studied approach to bidirectional programming, summarized in
Figure~\ref{fig:asymmetric-lens-def}, is the framework of \emph{asymmetric},
\emph{state-based} lenses. In this model, one of the two repositories is a
% TODO: definitely, definitely need some citations
\emph{view} of or \emph{query} on the other: that is, it can be completely
reconstructed from the other without additional outside information in case
it is lost. Calling the types of the two repositories $S$ (for ``source'')
and $V$ (for ``view''), we reflect this fact by requiring one of the
components of such a lens to be a function
\[\aget \in S \to V.\]
In most cases, a query will keep only some of the information available in
the source; as a result, the opposite reconstruction property---that the
source can be completely reconstructed from the view---usually does not
hold. Asymmetric, state-based lenses handle this situation by allowing their
other major function component to have access to both a modified value from
the view repository \emph{and} an original value from the source repository
to merge the new data into:
\[\aput \in V \times S \to S\]
As a technical detail, it is convenient to demand (and not difficult to
satisfy this demand in practice) a way to generate a value in the source
repository with some sane defaults. The most common presentation of this is
to have a third and final lens component:
\[\acreate \in V \to S\]
\begin{figure}
    \begin{minipage}{0.5\linewidth}
        \begin{align*}
            \aget &\in S \to V \\
            \aput &\in V \times S \to S \\
            \acreate &\in V \to S
        \end{align*}
    \end{minipage}
    \begin{minipage}{0.5\linewidth}
        \begin{align*}
            \aput(\aget(s),s) &= s \\
            \aget(\aput(v,s)) &= v \\
            \aget(\acreate(v)) &= v
        \end{align*}
    \end{minipage}
    \caption{The required components and behavioral guarantees of a lens in
    $S \alens V$}
    \label{fig:asymmetric-lens-def}
\end{figure}

\section{Contributions}
\label{sec:contributions}
\begin{itemize}
    \item We have made some valuable progress on each of these issues
        \begin{itemize}
            \item alignment: showed specific examples of lenses that
                disambiguate some especially confusing repository
                modifications
            \item give a natural way of factoring out the descriptions of
                extra global information -- treat it completely generically
                and see where that takes us
            \item symmetry: showed how to design a lens framework that
                supports composition and symmetry
            \item size: designed a series of lenses where the computations
                operate on small edits, rather than the large repository
        \end{itemize}
    \item We implemented xxx, which demonstrates yyy
    \item Hint at the fact that there is more to be done if you want to
        operate on multiple repositories simultaneously
\end{itemize}
