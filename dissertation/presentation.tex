\documentclass{beamer}
% {0,51,128} is the Penn blue color; this must come before
% \usepackage{beamer_minimal} so that the progress bar inherits the right
% color; it should be possible to make this independent but I don't have the
% time to learn the ins and outs of beamer colors
\usecolortheme[RGB={0,51,128}]{structure}
\usepackage{beamer_minimal}
\title{Generalizing Lenses}
\subtitle{A New Foundation for Bidirectional Programming}
\author{Daniel Wagner} % for hyperref
\date{June 13, 2014}

\begin{document}

\author{Daniel Wagner\\[3ex]\includegraphics[width=0.2\linewidth]{plclub-logo.pdf}} % for beamer
\maketitle

\end{document}
outline proposal
-  5: background + motivation
    - keep two pieces of data in synch, even though they are stored in a
      different format or have slightly different ``stuff'' inside
    - design transformation pairs in tandem
    - perhaps a bit of history? databases+complements; lenses and subsequent
      bloom of research; dissatisfaction with laws/asymmetry/alignment
-  5: scope of my work + the thesis + my contributions
    - contributions:
        - framework that supports symmetry, alignment, performance
        - algebraic study of the framework + an inhabiting syntax
        - prototype implementation
- 30: deep drill into something technical and cool; possibilities:
    - containers + container mapping + how alignment is solved with the edit
      language for containers; perhaps leading to container reshaping lens;
      perhaps coming from discussion of why recursive types are hard to
      design edit languages for
        - skip the lenses, talk just about edit languages to begin with
        - (free) edit monoid for products
        - (free) edit monoid for sums
        - make a couple attempts at an edit monoid for recursive types
        - notice that what we really want is some concept of a pointer into
          these structures
        - so model it that way explicitly, use containers
        - (free) edit monoid for containers
        - mapping lens for containers
        - observe how alignment is handled
        - discuss reshaping lens
    - behavioral laws: why the old ones don't cut it and how our
      monoid-based ones help; observation that partition is tricky and may
      require lax laws and what this tells us about the partition lens
    - the category of (edit) lenses, the machinery needed to make it be a
      category, what structure you get and don't (and why), what open
      questions this settles/insight this gives
    - possibly: the monoid isomorphism route to lenses (i.e. start with
      isomorphisms, see why they are too strong, see how to relax them, then
      add complements; can also discuss the ``one module per type''
      philosophy along the way to motivate adding complements vs. choosing a
      different edit language when designing partition lens)
-  5: pop back up a level and give some perspective + summary
    - symmetric lenses: first framework to offer symmetry + serious study of
      composition and the associated machinery
    - edit lenses: add incremental operation, talk about the processing of
      alignment information
    - implementation: still need some theory about generating alignment
      information

BCP says:
    * give some motivation/perspective... but make it quick; almost
      everybody in the audience has seen my talks before, so they know
      what's going on
    * plan to make it to the list of contributions within about five minutes
    * two purposes for the presentation; should heavily weight my efforts
      towards really satisfying the second goal
        * as a public announcement of the work I've done and contributions
          I've made
        * gives the committee a chance to solidify their understanding and
          opinions of my work
SCW says:
    * BUT the focus should be on what I have contributed; the
      history/perspective should be there so that it's clear what I've
      contributed to the historical understanding, the technical stuff
      should be there so they understand what I've contributed, etc.; the
      technical content isn't the focus, but a means to conveying the focus
    * also, re-expressed the preference for explaining one thing well, i.e.
      it's reasonable to skimp on old work in favor of a good, in-depth
      explanation of the newest work
    * she mentioned in passing something like, "maybe half the talk is on
      your newest technical work, and five minutes is on how that connects
      with older work"
SAZ says:
    * definitely reiterate the motivation for the work
    * explain contributions to someone already familiar with the work
    * very important: situate this work compared to other work
    * might want to give the committee an overview of what they asked for at
      the proposal and how I addressed those recommendations
SAZ suggested a rough timing outline:
    *  5min background + motivation
    *  5min scope of my work + the thesis + my contributions
    * 30min deep drill into something technical and cool
    *  5min pop back up a level and give some perspective + summary
