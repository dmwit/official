% high-level comments from BCP 20140117 (before anything was written):
% * on purpose: also just want to give an overview of the field as a whole
%   (in addition to just saying where we do better)
% * on writing: should give some exposition/explanation of the high-priority
%   papers (which are the category-based delta lenses and the algebraic
%   studies)
% * on citations: cite many, but discuss only the latest one in detail
\section{Category-based delta lenses}
\label{sec:delta}

\subsection{Asymmetric}
\label{sec:delta:asymmetric}
% describe ``From state-based to delta-based bidirectional transformations:
% the asymmetric case''

\subsection{Symmetric}
\label{sec:delta:symmetric}
% describe ``From state-based to delta-based bidirectional transformations:
% the symmetric case''

\subsection{Comparison}
\label{sec:delta:comparison}
% compare delta lenses to edit lenses
% do tackle symmetry; do have edit-based things; do have some neat new laws
% that we didn't consider; don't consider size; don't provide syntax
%
% some text below that we can also scan for ideas about what to write here

\section{Algebraic rephrasing}
\label{sec:algebraic}
% Perdita's explorations figuring out what properties transformations have
% without working out what those transformations are
% (but totally uninterested in syntax)
% TODO: use S and V instead of X and Y, to emphasize the connection with
% asymmetric lenses?
There is a line of work on algebraic foundations for delta lenses that
arrives at a model very similar to the edit lens framework described
above~\cite{stevens2008tat}. They consider, as we do, edit monoids together
with edit translation morphisms and (total) edit application actions.
One significant difference is that they consider generalizing asymmetric
rather than symmetric lenses, adopting correspondingly modified behavioral
laws. In particular, in their setting, a delta lens is a \emph{lens-like
split short exact sequence}. Below we discuss each of these restrictions in
right-to-left order. We begin with two standard definitions to fix some
notation.

\begin{definition}
    The \emph{image} of a function $f \in X \to Y$ is the set of elements
    $\im(f) \subset Y$ that $f$ can output:
    \[\im(f) = \{f(x) \mid x \in X\}\]
\end{definition}

\begin{definition}
    The \emph{kernel} of a monoid morphism $f$, denoted $\ker(f)$, is the
    preimage of $\ONE$:
    \[\ker(f) = \{x \mid f(x) = \ONE\}\]
\end{definition}

\begin{definition}
    An \emph{exact sequence} is a sequence $\blist f_1 \clist f_n \elist$ of
    monoid morphisms with compatible domains and codomains, that is,
    \begin{diagram}
        \path
            node                      (0)    {$X_0$}
            node[right=2.5em of 0]    (1)    {$X_1$}
            node[right=2.5em of 1]    (2)    {$X_2$}
            node[right=2.5em of 2]    (dots) {$\cdots$}
            node[right=2.5em of dots] (n2)   {$X_{n-2}$}
            node[right=2.5em of n2]   (n1)   {$X_{n-1}$}
            node[right=2.5em of n1]   (n0)   {$X_n$}
            (0)    edge[->] node[above] {$f_1$}     (1)
            (1)    edge[->] node[above] {$f_2$}     (2)
            (2)    edge[->] node[above] {$f_3$}     (dots)
            (dots) edge[->] node[above] {$f_{n-2}$} (n2)
            (n2)   edge[->] node[above] {$f_{n-2}$} (n1)
            (n1)   edge[->] node[above] {$f_n$}     (n0)
        ;
    \end{diagram}
    and such that $\im(f_i)=\ker(f_{i+1})$ for each $i$.
\end{definition}

\begin{definition}
    An exact sequence is \emph{short} if it has four morphisms and starts
    and ends at $\D\Unit$:
    \begin{diagram}
        \path
            node                   (1-1) {$\D\Unit$}
            node[right=4em of 1-1] (K)   {$K$}
            node[right=4em of K]   (X)   {$X$}
            node[right=4em of X]   (Y)   {$Y$}
            node[right=4em of Y]   (1-2) {$\D\Unit$}
            (1-1) edge[->] node[above] {$s$} (K)
            (K)   edge[->] node[above] {$k$} (X)
            (X)   edge[->] node[above] {$f$} (Y)
            (Y)   edge[->] node[above] {$i$} (1-2)
            ;
    \end{diagram}
    We will say \emph{around} to mean the third element of a sequence, as
    in, ``$\blist s \mlist k \mlist f \mlist i \elist$ is a short exact
    sequence around $f$.''
\end{definition}

Before we define what split and lens-like mean, let us consider when an
edit translation morphism $f \in X \to Y$ may be extended to a short exact
sequence. The morphism $i \in Y \to \D\Unit$ must be the constantly-$\ONE$
function (there are no other functions with that type), so that its kernel
is $\ker(i)=Y$. Hence the restriction $\im(f)=\ker(i)$ that arises from
extending the sequence to the right says that $f$ must be surjective. On the
other hand, the sequence may always be extended to the left by choosing $K$
to be the submonoid $\ker(f)$ and $k$ to be the inclusion function. (The
morphism $s$ is completely determined by the morphism laws once we have
chosen a monoid $K$: it must map the sole input element $\ONE_{\D\Unit}$ to
$\ONE_K$.) Other choices for $K$ are possible---for example, by adding a
fresh generator to $K$ that $k$ maps to any non-trivial element of $X$---but
we will not be interested in this ability below.

In lens terms, one should think of $f$ as being an edit-lens analog of the
asymmetric lens framework's \GET function. Giving a short exact sequence
amounts to identifying an edit translation function $f \in X \to Y$ that is
compatible with the monoid structure on edits and such that each $Y$-edit
has at least one analogous $X$-edit.

\begin{definition}
    A short exact sequence around $f \in X \to Y$ is said to \emph{split} if
    there is a morphism $g \in Y \to X$ such that $g;f = \id_Y$.
\end{definition}

We are guaranteed that there is a \emph{function} $g$ by the fact that $f$
is surjective, but not guaranteed that any such function is a monoid
morphism. If we do have such a morphism $g$ that splits the sequence, then
in lens terms we should consider that $g$ to be an edit-lens analog of the
\PUT function. Then $g;f = \id_Y$ says that all the information available in
$Y$-edits are also available in $X$-edits, an analogous restriction to the
one on state-based lenses that says that all the information available in
the view repository is available in the source repository.

The final condition placed on this variant of delta lenses is that they be
lens-like. Thus far, all the conditions have been purely in terms of edits;
this final pair of properties connect the world of edits and the world of
states in a way similar to the edit lens law that requires $\dputr$ and
$\dputl$ to respect a consistency relation.

\begin{definition}
    A monoid action $\odot \in \DX \times X \to X$ is \emph{transitive} if
    for all $x,x' \in X$ there is $\dx \in \DX$ such that $\dx\odot x=x'$. We
    will say a module is transitive when its action is.
\end{definition}

% TODO: this is so imprecise now, because we're using X instead of \DX
\begin{definition}
    A short exact sequence around $f \in X \to Y$ split by $g$ is
    \emph{lens-like} if it comes equipped with transitive, total modules for
    $X$ and $Y$ such that two conditions hold:
    \infrule[LL1]
        {\dx\;\init_X = \dx'\;\init_X}
        {f(\dx)\;\init_Y = f(\dx')\;\init_Y}
    \infrule[LL2]
        {\dy\;f(\dx)\;\init_Y = f(\dx)\;\init_Y}
        {g(\dy)\;\dx\;\init_X = \dx\;\init_X}
\end{definition}

% TODO: notation correspondence
% our \D\Unit is their 1
% our \init is their \Omega
% we use f and g for a split sequence; they use \mu and \lambda
% we use \dx and \dy (\d s and \d v?) for edits, they use g

\section{Matching lenses}
\label{sec:matching}
% defining good alignment heuristics along with your lens
% decouples alignment discovery from lens operation
% does give concrete operations for discovering alignments; but is
% significantly more complex; can only handle containers with a single kind
% of data inside (e.g. could give good treatment to either the [a] or the
% [b] in a ([a], [b]))

\section{Hole-based delta lenses}
\label{sec:holes}
% lenses that operate on full data structures with node
% creation/deletion/modification ``holes'' in them -- tackles alignment
% problems, but not size or symmetry
%
% some text below that we can also scan for ideas about what to write here

\section{Fibrations}
\label{sec:fibrations}
% Johnson+Rosebrugh's work on sketch-based data modeling; see FogBugz for a
% list of resources to consult when it comes time to write this section
%
% still digesting, but it seems they have a notion of edit but that it
% doesn't help with alignment (?) and that they don't really propose much in
% the way of syntax -- they only give a way to check if a particular
% hand-constructed transformation is allowed

\section{Constraint maintainers}
\label{sec:constraint-maintainers}
% symmetric, but we improve in many ways
%
% some text below that we can also scan for ideas about what to write here

\section{Others}
\label{sec:other-related-work}
% In case of time crunch, may be able to drop this section!

\subsection{Graph transformations}
\label{sec:other-related-work:graph}
% BiG-lab's stuff -- check if they make any attempt at symmetry or edits

\subsection{Triple graph grammars}
\label{sec:other-related-work:triple}
% gives an attempt at parsing graphs; some of the tools are symmetric, and
% may be in use, but need to review a bit before knowing what to say here

\subsection{Incremental XML transformations}
\label{sec:other-related-work:xml}
% sort of tangentially related; lots of stuff is based on tree transducers;
% very interested in good models for edits and edit transformations so
% perhaps we have something to teach them there

% dump from our delta lenses paper {{{
%
%The most closely related attempt at developing a theory of update
%propagation is \cite{Diskin-Delta11} by Diskin et al. Their starting
%point is the observation (also discussed in \cite{Matching10}) that discovery of
%edits should be decoupled from their propagation. They thus propose a
%formalism, \emph{sd-lenses}, for the propagation of edits across
%synchronized data structures, bearing some similarities with our
%edit lenses. The replicas, which we model as modules, are there modeled
%as categories (presented as reflexive graphs)\iffull, as we discussed in 
%\S \ref{cats}\fi. Thus, for any two states $x,x'$ there is a set of
%edits $X(x,x')$. An sd-lens then comprises two reflexive graphs $X,Y$
%and for any $x\in X$ and $y\in Y$ a set $C(x,y)$ of
%``correspondences'' which roughly correspond to our
%complements. Forward and backward operations similar to our $\dputl$ and
%$\dputr$ then complete the picture. No concrete
%examples are given of sd-lenses, no composition, no notion of equivalence, and
%no combinators for constructing sd-lenses; the focus of the paper is
%rather on the discovery of suitable axioms, such as invertibility and
%undoability of edits, and a 
%generalization of {\em hippocraticness} in the sense of
%Stevens~\cite{Stevens07}. They also develop a comparison 
%with the state-based framework (cf.\ \S \ref{stateb} above). In
%our opinion, the separation of edits and correspondences according to
%the states that they apply to or relate has two important
%disadvantages.  First, in our examples, it is often the case that one
%and the same edit applies to more than one state and can be
%meaningfully propagated (and more compactly represented) as such. For example, while many of the
%container edits tend to only work for a particular shape, they are
%completely polymorphic in the contents of the container. Second, the
%fact that state sets are already categories suggests 
%that a category of sd-lenses would be
%2-categorical in flavor, entailing extra technical difficulties such as
%coherence conditions. 
%
%%% \finish{Here's a quick start from a brief scan of their most recent
%%%   paper~\cite{Diskin-Delta11} by BCP...
%%%   \begin{itemize}
%%%   \item Their motivations and goals are exactly the same.
%%%   \item The technicalities of their approach are pretty dense.  I haven't
%%%   internalized them yet.  In particular, I don't have a good intuition for
%%%   their ``sameness'' relations---what they say makes sense for ``flat''
%%%   structures like lists or simple graphs, but not for more structured data
%%%   (where you'd want to know about correspondences at various levels of
%%%   structure). 
%%%   \item They don't say anything about the size of their deltas, and at
%%%   least a naive representation would be big.  We're much more careful about
%%%   this. 
%%%   \item They propose two new laws (weak invertability and undoability).  I'm
%%%   not sure what to say about these, but I guess it's an important point of
%%%   comparison, since it's one of the main points of their paper.
%%%   \item They don't handle $\missing$ (a small point)
%%%   \item They don't define any combinators, just the semantic space itself (a
%%%   larger point, and related) 
%%%   \item Their Definition 19 and Theorem 5 relate their delta-lenses to our
%%%   symmetric lenses.  I'm not completely sure how to interpret it (does our
%%%   ``trivial module'' correspond to their ``simple graph''?), but in any case
%%%   our result is stronger because it goes both directions.  (Their
%%%   characterization of our symmetric lenses is a little bit wrong---it puts
%%%   $C$ in the wrong place---but I'm not sure this matters.)
%%%   \end{itemize}
%%% }
%
%Meertens's seminal paper on {\em constraint maintainers}~\cite{Meertens98}
%discusses a form of containers for lists equipped with a notion of edits
%similar to our edit language for lists, but does not develop a general
%theory of edit-transforming constraint maintainers.
%
%A long series of papers from the group at the University of Tokyo
%\cite[etc.]{Hu04, Mu2004, MuAlgebraic2004, HuModels07,
%  Hidaka10}\iffull\discuss{double-check these, and add more, and add to abstract
%  too}\fi{} deal with the alignment issue using an approach that might be
%characterized as a hybrid of state-based and edit-based.  Lenses work with
%whole states, but these states are internally annotated with tags showing
%where edits have been applied---e.g., marking inserted or deleted elements
%of lists.
%%
%Barbosa et al.'s {\em matching lenses}~\cite{Matching10} offer another 
%approach to dealing with issues of alignment in the framework of pure
%state-based lenses.  
%
%
%%% \noindent
%%% Other things we definitely need to compare to:
%%% \begin{itemize}
%%% \item Diskin's ``tile algebras'' \cite{DBLP:conf/gttse/Diskin09}
%%% \item The other delta-lens papers that Diskin refers to in the intro
%%% of~\cite{Diskin-Delta11} 
%%% \item Other papers addressing the alignment problem in different ways (e.g.,
%%% our Matching Lenses~\cite{Matching10}, Tokyo group papers such
%%% as~\cite{Mu2004} and maybe the ICFP10 paper on bidirectional graph
%%% transformations)
%%% \item Maybe some papers on edits in the context of version management (see
%%% last year's grant proposal for some citations)
%%% \item other lens-like things that have used some kind of deltas~\cite{HuModels07} 
%%% \begin{itemize}
%%% \item \finish{Meertens \cite{Meertens98} introduces edit operations in section 5.3
%%% (p. 68ff) to talk about edits to lists.  I have not grokked yet exactly how
%%% all this works, or how it fits into his general framework of constraint
%%% maintainers.  Some possible text:}
%
%%% \item 
%%% \end{itemize}
%%% \end{itemize}
%
%%% \iflater
%%% \finish{
%%% Other things to think about:
%%% \begin{itemize}
%%%     \item other research on edit lenses
%%%     \item practical tools that use notions of an edit
%%%     \item Is there any correspondence with other tools like SVN, Unison,
%%%     ...? 
%%% \item operation transform papers?
%%% \end{itemize}
%%% }
%%% \fi
%
% }}}

% dump from our complement lenses paper {{{
%
%There is a large literature on lenses and related approaches to
%propagating updates between connected structures.  We discuss only the most
%closely related work here; good general surveys of the area can be found
%in~\cite{FosterThesis,DBLP:conf/icmt/CzarneckiFHLST09}.  Connections to the
%literature on {\em view update} in databases are surveyed
%in~\cite{Focal2005-shortcite}. \iffull A short version of this paper is available
%in~\cite{HofmannPierceWagner10}.\fi
%
%The first symmetric approach to update propagation was proposed by
%Meer\-tens~\cite{Meertens98} and followed up \iffull in the context of
%model-driven design \fi by Stevens~\cite{Stevens07},
%Diskin~\cite{DBLP:conf/models/Diskin08}, and Xiong, et
%al~\cite{xiong2009supporting}.
%%
%Meertens suggests modeling synchronization between two sets
%$X$ and $Y$ by a {\em consistency relation} $R\subseteq
%X\times Y$ and two {\em consistency maintainers}
%$\triangleleft: X\times Y\rightarrow X$ and $\triangleright: X\times Y
%\rightarrow Y$ such that $(x\triangleleft y) \relR y$ and
%$x \relR (x\triangleright y)$ always hold, and such that $x \relR y$ implies
%$x \triangleleft y = x$ and $x \triangleright y = y$.
%
%The main advantage of symmetric lenses over consistency maintainers is
%their closure under composition. Indeed, all of the aforementioned
%authors note that, in general, consistency maintainers do not compose
%and view this as a drawback.
%%
%Suppose that we have relations $R\subseteq X\times Y$ and
%$R'\subseteq Y\times Z$ maintained by $\triangleright,\triangleleft$
%and $\triangleright', \triangleleft'$, resp. If we want to construct a
%maintainer for the composition $R;R'$, we face the problem that, given
%$x\in X$ and $z\in Z$, there is no canonical way of coming up with a
%$y\in Y$ that will allow us to use either of the existing maintainer
%functions. Concretely, Meertens gives the following counterexample.
%Let $X$ be the set of nonempty context free grammars over some alphabet, and let
%$Y$ be the set of words over that same alphabet. Let $R\subseteq
%X\times Y$ be given by $G \relR x\iff x\in L(G)$. It is easy to define
%computable maintainer functions making this relation a constraint
%maintainer. Composing this relation with its opposite yields an 
%undecidable relation (namely, whether the intersection of two context-free
%grammars is nonempty), so there cannot be computable maintainer functions.
%
%We can transform any constraint maintainer into a  symmetric lens as
%follows: take the relation $R$ itself (viewed as a set of pairs) as
%the complement, and define $\putl(x',(x,y))=(x'\triangleright
%y,(x',x'\triangleright y))$ and similarly 
%for $\putr$. If we compose such a symmetric lens with its opposite
%we obtain $R\times R\op$ as the complement and, for example,
%$\putr(x',((x_1,y_1),(y_2,x_2))) = 
%(x_2\triangleleft(x'\triangleright y_1), ((x',x'\triangleright
%y_1),(x'\triangleright y_1,x_2\triangleleft(x'\triangleright y_1))))$. 
%%
%For Meertens' counterexample, we would have complements of the form
%$((G_1,w_1),(w_2,G_2))$, with $w_1\in L(G_1)$
%and $w_2\in L(G_2)$; ``$\putr$''-ing a new grammar
%$G_1'$ through the composed lens yields the complement
%$((G_1',w_1'),(w_1',G_2'))$, where $w_1'$ is $w_1$ if $w_1\in L(G_1)$ and
%some default otherwise, and where $G_2'=G_2$ if $w_1'\in L(G_2)$ and
%$S{\rightarrow}w_1'$ (where $S$ is the start state) otherwise. We observe
%that there is a property of lenses analogous to Meertens' requirement that
%$x \relR y$ implies $x \triangleleft y = x$. This property is not
%necessarily preserved by composition, and in particular the lens described
%above for synchronizing languages does not have it.
%%
%Meertens recommends using a {\em chain} of consistency maintainers in such a
%situation to achieve a similar effect; however, the properties of such
%chains have not been explored.
%
%%%%%%%%%%%%%%%%%%%%%%%%%%%%%%%%%%
%
%For asymmetric lenses, a number of alternative \iffull choices of behavioral
%\fi
%laws 
%have been explored.  Some of these are \iffull strictly \fi weaker than ours; for
%example, a number of papers from a community of researchers based in Tokyo
%replace the \rn{PutGet} law with a somewhat looser \rn{PutGetPut} law,
%permitting a broader range of useful behaviors for lenses that duplicate
%information.  It would be interesting to see what kind of categorical
%structures arise from these choices.  The proposal by Matsuda et
%al.~\cite{matsuda2007btb} is particularly interesting because it also employs
%the idea of complements.  Conversely, stronger laws can be imagined, such as
%the \rn{PutPut} law discussed by Foster et
%al.~\cite{Focal2005-shortcite}\iffull{} and the more refined variants
%in~\cite{updatable-security-views}\fi\iflater\finish{Say something more
%  about this?}\fi.
%
%A different foundation for defining lenses by recursion was explored by
%Foster et al.~\cite{Focal2005-shortcite}, using standard tools from domain
%theory to define monotonicity and continuity for lens combinators
%parametrized on other lenses.  The main drawback of this approach is that
%the required (manual) proofs that such recursive lenses are total tend to be
%somewhat intricate.  By contrast, we expect that our initial-algebra
%approach can be equipped with automatic proofs of totality (that is, choices
%of the weight function $w$) in many cases of interest.
%
%\ifdelta
%\finish{
%Updates: (note that the point in our paper is not update as such but rather
%how to marry it with lenses)  \finish{BCP will look for a good canonical
%  reference for updates}
%\begin{itemize}
%\item lens-like things that have used deltas~\cite{Hu04,Meertens98,MuAlgebraic2004,HuModels07}
%\item old update transformation
%\item old view update stuff
%\item new Waterloo stuff 
%\begin{itemize}
%\item flexibility, not efficiency
%\item no proposal for syntax---composition is the only operator they study
%at all
%\end{itemize}
%\end{itemize}
%}
%\fi
%
% }}}
