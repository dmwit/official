% TODO: cleanup
\section{Background}
\newif \iftext  \texttrue
\newif \iffull  \fulltrue
\newif \ifdraft \draftfalse
\newif \ifdelta \deltafalse
\newif \iflater \laterfalse  % (for things that we're going to think about later)
\newif \iftikz  \tikztrue
To set the stage, let's review the standard definition of
asymmetric lenses.  
%
(Other definitions can be given, featuring weaker or stronger laws, but this
version is widely accepted.)
%
Suppose $X$ is some set of source structures (say, the possible states of a
database) and $Y$ a set of target structures (views of the database).
%
An asymmetric lens from $X$ to $Y$ has two components:
\[ 
\begin{array}{r@{\ \;}c@{\ \;}l}
\GET &\in& X \arrow Y\\
\PUT &\in& Y \times X \arrow X
\end{array}
\]
The \GET{} component is the forward transformation, a total function from
$X$ to $Y$.  The \PUT{} component takes an old $X$ and a modified $Y$ and
yields a correspondingly modified $X$.  These components
must obey two ``round-tripping'' laws for every $x \in X$ and $y
\in Y$:
%
\infax[GetPut]{
  \PUT\; (\GET \; x)\; x = x 
}
\infax[PutGet]{
  \GET\; (\PUT \; y \; x) = y  
}
%
It is also useful to be able to create an element of $x$ given just an
element of $y$, with no ``original $x$'' to put it into; in order to handle
this in a 
uniform way, each lens is also equipped with a
function $\CREATE\in Y\arrow X$, and we assume one more axiom:
\infax[CreateGet]{
    \GET\; (\CREATE \; y) = y
}

\section{Symmetric lenses}
First, we show that lenses can be generalized from their usual asymmetric
presentation---where one of the structures is always a ``view'' of the
other---to a 
fully {\em symmetric} version where each of the two structures may 
contain information that is not present in the other.
This generalization is significantly more expressive than any previously
known: although symmetric variants of lenses have been
studied~\cite{Meertens98,stevens2008tat,DBLP:conf/models/Diskin08}, they
all lack a notion of sequential composition of lenses, a significant
technical and practical limitation.
%
As we will see, the extra structure that we need to support composition is
nontrivial; in particular, constructions involving
symmetric lenses need to be proved correct modulo a notion of {\em behavioral
  equivalence}.

Second, we undertake a systematic investigation of the {\em
  algebraic structure} of the space of lenses, using the concepts of
elementary category theory as guiding and organizing principles.  (Our
presentation is self contained, but some prior familiarity
with basic concepts of category theory will be helpful.)

The key  step toward symmetric lenses is the notion
of {\em complements}.  The idea dates back to a famous paper in the
database literature on the view update
problem~\cite{DBLP:journals/tods/BancilhonS81} and was adapted to
lenses in~\cite{Matching10} (and, for a slightly different
definition,~\cite{matsuda2007btb}), and it is quite simple.  If we think of
the    
\GET{} component of a lens as a sort of projection function, then we can find
another projection from $X$ into some set $C$ that
keeps all the information discarded by \GET{}.  Equivalently, we can think
of \GET{} as returning two results---an element of $Y$ and an element of
$C$---that together contain all the information needed to reconstitute the
original element of $X$.  Now the \PUT{} function doesn't need a whole $x\in
X$ to recombine with some updated $y\in Y$; it can
just take the complement $c\in C$ generated from $x$ by the \GET, since this
will 
contain all the information that is missing from $y$.  Moreover, instead of
a separate
$\CREATE$ function, we can simply pick a distinguished element
$\missing\in C$ and define $\CREATE(y)$ as $\PUT(y,\missing)$.

Formally, an {\em asymmetric lens with complement}
mapping between $X$ and $Y$ consists of a set $C$, a
distinguished element $\missing \in C$, and two functions
\[
\begin{array}{r@{\ \;}c@{\ \;}l}
\GET &\in& X \arrow Y \times C\\
\PUT &\in& Y \times C \arrow X
\end{array}
\]
obeying the following laws for every $x \in X$, $y \in Y$, and $c \in C$:%
\footnote{We can convert back and
forth between the two presentations; in particular, if $(\GET, \PUT,
\CREATE)$ are the components of a traditional lens, then we define a
canonical complement by $C=\{f\in Y{\rightarrow}X\mid \forall
y.\;\GET(f(y))=y\}$. We then define the components $\missing'$, $\GET'$, and
$\PUT'$ 
of an asymmetric lens with complement as $\missing'=\CREATE$ and
$\GET'(x)=(\GET(x),\lambda y.\PUT(y,x))$ and $\PUT'(y,f)=f(y)$.
\iffull
Going the other way, if $(\aget, \aput, \missing)$ are the components of an
asymmetric lens with complement, we can define a traditional lens by
$\aget'(x) = \mlfst(\aget(x))$ and $\aput'(y,x) = \aput(y,\mlsnd(\aget(x)))$
and $\acreate(y) = \aput(y,\missing)$.
\fi
}
%
\infrule[GetPut]{
  \GET\; x = (y,c)
}{
  \PUT\; (y,c) = x 
}
\infrule[PutGet]{
  \GET\; (\PUT \; (y,c)) = (b',c')
}{
  b' = y
}
Note that the type is just ``lens from $X$ to $Y$'': the set
$C$ is an internal component, not part of the externally visible type.
In symbols, $ \mathit{Lens}(X,Y) =
\exists C.\; \{ \mathord{\missing}:\, C,\; \GET:\,X \arrow Y
\times C,\; \PUT:\,Y \times C \arrow X \}$.

Now we can symmetrize.  
First, instead of having only \GET{} return a complement, we make \PUT{}
return a complement too, and we take this complement as a second argument
to $\GET$.
%
\iffull
\[
\begin{array}{r@{\ \;}c@{\ \;}l}
\GET &\in& X \times C_Y \arrow Y \times C_X\\
\PUT &\in& Y \times C_X \arrow X \times C_Y
\end{array}
\]
\else
So we have $\GET \in X \times C_Y \arrow Y \times C_X$ and 
$\PUT \in Y \times C_X \arrow X \times C_Y$.
\fi
%
Intuitively, $C_X$ is the ``information from $X$ that is discarded by
\GET,'' and $C_Y$ is the ``information from $Y$ that is discarded by
\PUT.''  Next we observe that we can, without loss of generality, use the
same set $C$ as the complement in both directions.  \iffull (This ``tweak''
is actually critical: it is what allows us to define composition of symmetric
lenses.)\fi
\iffull
\[
\begin{array}{r@{\ \;}c@{\ \;}l}
\GET &\in& X \times C \arrow Y \times C\\
\PUT &\in& Y \times C \arrow X \times C
\end{array}
\]
\else
So now we have 
$\GET \in X \times C \arrow Y \times C$ and
$\PUT \in Y \times C \arrow X \times C$.
\fi
%
Intuitively, we can think of the combined complement $C$ as $C_X \times
C_Y$---that is, each complement contains some ``private information from
$X$'' and some ``private information from $Y$''; by convention, the \GET{}
function reads the $C_Y$ part and writes the $C_X$ part, while the \PUT{}
reads the $C_X$ part and writes the $C_Y$ part.  
%
Lastly, now that everything is symmetric, the \GET{} / \PUT{} distinction is not
helpful, so we rename the functions to \PUTR{} and \PUTL.  This brings us to
our core definition.

\iftikz
\iffull
\begin{figure*}[t!] \centering
\vspace*{-4ex}
\hspace*{-1em}
\begin{tabular}{@{}cc}
  \ifpdf\tikz\pdf{symmetric-minus};\vspace*{-1ex}
  \else \tikz\pdf{symmetric-minus}node[below=8.2ex]{};
  \fi
 &
  \tikz\pdf{symmetric};
  \ifpdf\vspace*{-2ex}\fi
 \\
(a) Initial replicas & (b) Initial complement 
\vspace*{4ex} \\
  \ifpdf\tikz\pdf{symmetric-edit};\vspace*{-.7ex}
  \else \tikz\pdf{symmetric-edit}node[below=10ex]{};
  \fi
&
  \tikz\pdf{symmetric-propagatex};
\\
(c) One replica edited & (d) Propagating the edit 
\vspace*{3ex}
\\
  \ifpdf\vspace*{3ex}\tikz\pdf{symmetric-edit2};\vspace*{-8ex}
  \else \tikz\pdf{symmetric-edit2}node[below=12.9ex]{};
  \fi
&
  \ifpdf\tikz\pdf{symmetric-propagate2};\vspace*{-1ex}
  \else \tikz\pdf{symmetric-propagate2}node[below=12.9ex]{};
  \fi
\\
(e) Second replica is edited & (f) This change is propagated
\vspace*{1.5ex}
\\
\end{tabular}
\caption{Behavior of a symmetric lens}\vspace*{2ex}
\label{fig:symm}
\end{figure*}
\else
\begin{figure*}[t!] \centering
\vspace*{-4ex}
\hspace*{-1em}
\begin{tabular}{@{}ccc}
  \ifpdf\tikz\pdf{symmetric-minus};\vspace*{-1ex}
  \else \tikz\pdf{symmetric-minus}node[below=8.2ex]{};
  \fi
 &
  \tikz\pdf{symmetric};
  \ifpdf\vspace*{-1ex}\fi
&
  \ifpdf\tikz\pdf{symmetric-edit};\vspace*{-3ex}
  \else \tikz\pdf{symmetric-edit}node[below=10ex]{};
  \fi
 \\
(a) Initial replicas & (b) Initial complement & (c) One replica edited 
\vspace*{2ex} \\
  \tikz\pdf{symmetric-propagatex};
&
  \ifpdf\vspace*{3ex}\tikz\pdf{symmetric-edit2};\vspace*{-4ex}
  \else \tikz\pdf{symmetric-edit2}node[below=12.9ex]{};
  \fi
&
  \ifpdf\tikz\pdf{symmetric-propagate2};\vspace*{-1ex}
  \else \tikz\pdf{symmetric-propagate2}node[below=12.9ex]{};
  \fi
\\
(d) Propagating the edit & (e) Second replica is edited & (f) This change is propagated
\vspace*{1ex}
\\
\end{tabular}
\caption{Behavior of a symmetric lens}
\label{fig:symm}
\end{figure*}
\fi
\fi

\begin{definition}[Symmetric lens]
A lens $\ell$ from $X$ to 
$Y$ (written $\ell \in X \lens Y$) has three parts:
a set of complements $C$, a distinguished element $\missing \in
C$, and two functions
\begin{eqnarray*}
    \putr &\in& X \times C \to Y \times C\\
    \putl &\in& Y \times C \to X \times C
\end{eqnarray*}
satisfying the following round-tripping laws:
\infrule[PutRL]{\putr(x,c) = (y,c')}{\putl(y,c') = (x,c')}
\infrule[PutLR]{\putl(y,c) = (x,c')}{\putr(x,c') = (y,c')}
When several lenses are under discussion, we use record notation to identify
their parts, writing $\ell.C$ for the complement set of $\ell$, etc. 
\end{definition}

\iftext The force of the \rn{PutRL} and \rn{PutLR} laws is to establish some
``consistent'' or ``steady-state'' triples $(x,y,c)$, for which \PUT{}s of $x$
from the left or $y$ from the right will have no effect---that is, will not
change the complement. The conclusion of each rule has the same variable
$c'$ on both sides of the equation to reflect this.  We will use the
equation $\putr(x,c) = (y,c)$ to characterize the steady states.  In
general, a \PUT{} of a new $x'$ from the left entails finding a $y'$ and a
$c'$ that restore consistency.  Additionally, we often wish this
process to involve the
complement $c$ from the previous steady state; as a result, it can be
delicate to choose a good value of $\missing$. This value can often be
chosen compositionally; each of our primitive lenses and lens combinators
specify one good choice for $\missing$.

\iflater\finish{There's a good technical discussion of the options for
  dealing with creation in symmetric.v --- might be worth including it
  here.}\fi \fi

\iffull\else One can imagine other laws.  In particular, the long version of
the paper considers symmetric forms of the asymmetric ``\rn{PutPut}'' laws,
which specify that two \PUT{} operations in a row should have the same
effect as the second one alone.  As with asymmetric lenses, these
laws appear too strong to be desirable in practice.  \fi

%% %
%% We say that lenses $l$ and $l'$ are {\em equivalent} if there exists a
%% relation $\mathord{\equivl} \subseteq C \times \C{l'}$ such that:
%% \infax[MissingC]{l.\missing \equivl l'.\missing}
%% \infrule[PutrC]{
%%   c \equivl c' 
%%   \andalso
%%   l.\PUTR (a, c) = (b,c_1)
%%   \andalso
%%   l'.\PUTR (a, c') = (b,c_1')
%% }{
%%   b = b' \ \wedge\   c_1 \equivl c_1'
%% }
%% \infrule[PutlC]{
%%   c \equivl c' 
%%   \andalso
%%   l.\PUTL (b, c) = (a,c_1)
%%   \andalso
%%   l'.\PUTL (b, c') = (a,c_1')
%% }{
%%   a = a' \ \wedge\   c_1 \equivl c_1'
%% }

%% With these definitions in place, we can develop some basic
%% structure.  For every set $A$, we can define is an identity lens $\mathit{id}$
%% (with a trivial complement) from $A$ to itself.  If we have a lens $l$
%% from $A$ to $B$ and a lens $l'$ from $B$ to $X$, we can compose them to form
%% a lens $(l;l')$ from $A$ to $X$, using $l.C \times l'.C$ as the complement.
%% %
%% We can also show that, up to equivalence, composition is associative and
%% \emph{id} is its unit---i.e., symmetric lenses form a category.  

%% \iffull \else \iftext \finish{Briefly mention the PutPut laws and the fact
%%   that, as usual, they are too strong.} \fi \fi
%% \iffull

\paragraph*{Examples} Figure~\ref{fig:symm} illustrates the use of a
symmetric lens.  
%
The structures in this example are lists of textual records describing
composers. The partially synchronized records (a) have a name and two dates
on the left and a name and a country on the right.
%
The complement (b) contains all the information that is discarded by both
$\PUT$s---all the dates from the left-hand structure and all the countries
from the right-hand structure.  (It can be viewed as a pair of lists of
strings, or equivalently as a list of pairs of strings; the way we build
list lenses later actually corresponds to the latter.)  If we add a
new record to the left hand structure (c) and use the $\PUTR$ operation to
propagate it through the lens (d), we copy the shared information (the new
name) directly from left to right, store the private information (the new
dates) in the complement, and use a default string to fill in both the
private information on the right and the corresponding right-hand part of
the complement.  If we now update the right-hand structure to fill in the
missing information and correct a typo in one of the other names
(e), then a $\PUTL$ operation will propagate the edited country to the
complement, propagate the edited name to the other structure, and use the
complement to restore the dates for all three composers.

Viewed more abstractly, the
connection between the information about a single composer in the two tables
is a lens from $X \times Y$ to $Y \times Z$, with complement $X
\times Z$---let's call this lens $e$.  Its \PUTR{} component is given $(x,y)$ as
input and has $(x',z)$ in its complement; it constructs a new complement by
replacing $x'$ by $x$ to form $(x,z)$, and it constructs its output by
pairing the $y$ from its input and the $z$ from its complement to form
$(y,z)$. The \PUTL{} component does the opposite, replacing the $z$ part of
the complement and retrieving the $x$ part.  Then the
top-level lens in Figure~\ref{fig:symm}---let's call it
$e\LIST$---abstractly has type $(X \times Y)\LIST \lens (Y \times Z)\LIST$
and can be thought of as the ``lifting'' of $e$ from elements
to lists.  

There are several plausible implementations of
$e\LIST$, with slightly different behaviors when list elements are
added and removed---i.e., when the input and complement arguments to \PUTR{}
or \PUTL{} are lists
of different lengths.  One possibility is to take $e\LIST.C = (e.C)\LIST$
and maintain the invariant that the complement list in the output is the same length as
the input list. When the lists in the input have different lengths, we can
restore the 
invariant by either truncating the complement list or padding it with
$e.\missing$.
% \iffull
For example, taking $X = \{a,b,c,\ldots\}$, $Y = \{1,2,3,\ldots\}$, $Z =
\{A,B,C,\ldots\}$, and $e.\missing = (m,M)$, and writing
$\left<a,b,c\right>$ for the sequence with the three elements $a$, $b$, and
$c$, we could have:
\[
\begin{array}{ll}
& \PUTR (\left<(a,1)\right>,\;
         \left<(p,P),(q,Q)\right>)
\\
= &\PUTR (\left<(a,1)\right>,\;
         \left<(p,P)\right>)\mbox{\hspace{7.7em}(truncating)}
\\
= & ( \left<(1,P)\right>,\; \left<(a,P)\right>)
\\ [1.5ex]
& \PUTR (\left<(a,1),(b,2)\right>,\;
         \left<(a,P)\right>)
\\
= &\PUTR (\left<(a,1),(b,2)\right>,\;
         \left<(a,P),(m,M)\right>)\mbox{\qquad(padding)}
\\
= & (\left<(1,P),(2,M)\right>,\;
         \left<(a,P),(b,M)\right>)
\end{array}
\]
% \fi
Notice that, after the first \PUTR{}, the information in the second
element of the complement list $(q,Q)$ is lost.
The second \PUTR{} creates a brand new second element for the list, so the value $Q$ is
gone forever; what's left is the default value $M$.

Another possibility---arguably better behaved---is to keep
  an {\em 
  infinite} list of complements.  Whenever we do a \PUT{}, we use (and
update) a prefix of the complement list of the same length as the current
value being \PUT, but we keep the infinite tail so that, later, we have
  values to use when the list being \PUT{} is longer.
%% \finish{Probably we should ``iffull'' the following display once it's been
%%   checked, to save space.}
% \iffull
\[
\begin{array}{ll}
& \PUTR (\left<(a,1)\right>,\;
         \left<(p,P),(q,Q),(m,M),(m,M),\ldots\right>) 
\\
= & ( \left<(1,P)\right>,\; \left<(a,P),(q,Q),(m,M),(m,M),\ldots\right>)
\\ [1.5ex]
& \PUTR (\left<(a,1),(b,2)\right>,\;
         \left<(a,P),(q,Q),(m,M),(m,M),\ldots\right>)
\\
= & (\left<(1,P),(2,Q)\right>,\;
         \left<(a,P),(b,Q),(m,M),\ldots\right>)
\end{array}
\]
% \fi

We call the first form the {\em forgetful} list mapping lens and the second
the {\em retentive} list mapping lens.  We will see, later, that the
difference between these two precisely boils down to a difference in the
behavior of the lens-summing operator $\oplus$ in the specification
$e\LIST \simeq \id_\Unit \oplus (e \otimes e\LIST)$ of the list mapping lens.

\iftikz
\begin{figure*}[t!] \centering
\vspace*{-4ex}
\begin{tabular}{@{}ccc}
  \tikz\pdf{sums1};
  &
  \tikz\pdf{sums2};
  \ifpdf\else\vspace*{2ex}\fi
  \\
  (a) Initial replicas & (b) Alphabetizing the right
  \vspace*{2ex} \\
  \tikz\pdf{sums3};
  &
  \tikz\pdf{sums4};
  \ifpdf\else\vspace*{2ex}\fi
  \\
  (c) Inserting Chopin on the left & (d) Deleting Beethoven from the left
\end{tabular}
\caption{Synchronizing lists of sums}
\label{fig:sums}
\end{figure*}
\fi
Figure~\ref{fig:sums} illustrates another use of symmetric lenses. The
structures in this example are lists of categorized data; each name on the
left is either a composer (tagged {\tt inl}) or an author (tagged
{\tt inr}), and each name 
on the right is either a composer or an actor.  The
lens under consideration will synchronize just the composers between the two
lists, leaving the authors untouched on the left and the actors untouched on
the right. The synchronized state (a) shows a complement with two lists,
each with holes for the composers.  If we re-order the
right-hand structure (b), the change in order will be
reflected on the left by swapping the two composers. Adding another composer
on the left
(c) involves adding a new hole to each complement; on the left, the location
of the hole is determined by the new list, and on the right it simply shows
up at the end. Similarly, if we remove a composer (d), the
final hole on the other side disappears.

Abstractly, to achieve this behavior we need to define a lens $\comp$
between $(X+Y)\LIST$ and 
$(X+Z)\LIST$.  To do this, it is convenient to first define a lens that
connects $(X+Y)\LIST$ and $X\LIST \times Y\LIST$; call this lens $\partition$.
The complement of the $\partition$ is a list of booleans telling whether the
corresponding element of the left list is an $X$ or a $Y$. The $\putr$
function is fairly simple: we separate the $(X+Y)$ list into $X$ and $Y$
lists by checking the tag of each element, and set the complement to exactly
match the tags. For example:
\begin{align*}
\putr(\left<\mlinl a,\mlinl b,\mlinr 1\right>,c) &=
    ((\left<a,b\right>,\left<1\right>),\left<\false,\false,\true\right>) \\
\putr(\left<\mlinl a,\mlinr 1,\mlinl b\right>,c) &=
    ((\left<a,b\right>,\left<1\right>),\left<\false,\true,\false\right>)
\end{align*}
These examples demonstrate that $\putr$ ignores the complement entirely,
fabricating a completely new one from its input. The $\putl$ function, on
the other hand, relies entirely on the complement for its ordering
information. When there are extra entries (not accounted for by the
complement), it adds them at the 
end. Consider taking the output of the second $\putr$ above and
adding $c$ to the $X$ list and $2$ to the $Y$ list:
\\[1.5ex]
\noindent\begin{tabular}{l}
$\putl((\left<a,b,c\right>,\left<1,2\right>),\left<\false,\true,\false\right>) =$ \\
\qquad$(\left<\mlinl a,\mlinr 1,\mlinl b,\mlinl c,\mlinr 2\right>,$ \\
\qquad$\left<\false,\true,\false,\false,\true\right>)$
\end{tabular}
\\[1.5ex]
\noindent The $\putl$ fills in as much of the beginning of the list as it
can, using the complement to indicate whether to draw elements from $X\LIST$
or from $Y\LIST$.  (How the remaining $X$ and $Y$ elements are interleaved
is a free choice, not specified by the lens laws, since this case only
arises when we are {\em not} in a round-tripping situation. The strategy
shown here, where all new $X$ entries precede all new $Y$ entries, is just
one possibility.)

Given $\partition$, we can obtain $\comp$ by composing three lenses in
sequence: from $(X+Y)\LIST$ we get to $X\LIST \times Y\LIST$ using
$\partition$, then to $X\LIST \times Z\LIST$ using a variant of the lens $e$ 
discussed above, and finally to $(X+Z)\LIST$ using a ``backwards''
$\partition$. 

\iffull

\section{Edit lenses}
% TODO
