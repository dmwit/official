\pagenumbering{roman}
\doublespaced
\large\newlength{\oldparskip}\setlength\oldparskip{\parskip}\parskip=.3in
\thispagestyle{empty}
\begin{center}
\vspace*{\fill}
\thetitle

\theauthor


A Dissertation Proposal

in

Computer Science
\end{center}


% XXX: phantoms below are to cut out stuff that should be in the thesis but
% not the proposal
\noindent\singlespaced\large
\phantom{Presented to the Faculties of the University of Pennsylvania in Partial
Fulfillment of the Requirements for the Degree of Doctor of Philosophy}


\doublespaced\large
\begin{center}
\theyear
\end{center}


\phantom{%
\noindent\makebox[0in][l]{\rule[2ex]{3in}{.3mm}}
\singlespaced
Advisor's Name\\
Supervisor of Dissertation}


\phantom{%
\noindent\makebox[0in][l]{\rule[2ex]{3in}{.3mm}}
\singlespaced
Graduate Chair's Name\\
Graduate Group Chairperson}
\vspace*{\fill}

\normalsize\parskip=\oldparskip


\newpage
\doublespaced

% TODO: fill in the acknowledgments section
%\chapter*{Acknowledgments}

\newpage
%\vspace*{\fill}
\begin{center}
  ABSTRACT\\
\thetitle\\
\vspace{.5in}
  \theauthor\\
  \theadvisor
\end{center}

\singlespaced
\noindent

Bidirectional situations are all around us: we synchronize bookmarks, keep
clones of our data on our mobile devices, edit collaboratively, use GUIs to
visualize and modify chunks of our data. In these situations, we must write
programs that translate back and forth between two data sets which
potentially use different storage formats and sometimes even record
differing characteristics of the data. The framework of lenses has been
% TODO: citation, maybe?
proposed as one tool to aid in the creation and maintenance of these
translations by giving a single program which can be interpreted as both a
forward and backward translation.

The basic lens framework, however, makes some assumptions about how the
transformations they describe will be used:
\begin{enumerate}
    \item There is a ``source repository'' which stores all of the data of
        interest and ``view repository'' which stores less data.
    \item It is reasonable to transmit and traverse entire repositories.
    \item There are two repositories of interest (or at least, if there are
        many repositories, they can reasonably be linked pairwise).
\end{enumerate}
For some decentralized applications, there are two repositories which share
some data, but which also each have their own unshared data; because there
is no master source repository, assumption (1) is violated. For large data
sets, it is reasonable to hope that one could store and transmit small
descriptions of recent changes to the repositories, which violates
assumption (2). Finally, some business applications can be phrased as having
many data repositories with intricate webs of connections between them,
violating assumption (3).

We will describe the generalized lens frameworks of symmetric lenses, which
relax assumption (1), and edit lenses, which relax assumption (2). We
propose a novel lens framework based on local constraint propagation which
relaxes assumption (3).
% TODO
% - don't talk about business applications, talk about spreadsheets
% - repositories is jargon
% - in general, make a pass to make sure that everything is understandable
%   to outsiders
% - add to the last paragraph: have one paragraph on what will be in the
%   dissertation, and one on what is in the proposal

\vspace*{\fill}

\newpage

\tableofcontents

\newpage
\draftspaced
\pagenumbering{arabic}
