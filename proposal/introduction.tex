As the electronic world grows increasingly interconnected, it grows
increasingly common to need good tools for synchronizing replicated data. In
many cases, the data stores being replicated are not identical---each store
is tailored to the device or application that is using the replica.
Traditional tools for maintaining synchrony between differing data formats
provide separate transformation tools for each pair of formats. However, as
the formats and transformations grow complex, maintaining separate tools can
grow more difficult and error-prone.

% TODO: definitely needs a citation here
The lens framework attempts to address this problem by giving a language of
transformations that can be interpreted \emph{bidirectionally}: a lens
between data formats $A$ and $B$ describes \emph{both} a transformation from
$A$ to $B$ \emph{and} a transformation from $B$ to $A$. Recent work has
developed some nice tooling for lenses, but there remain many use cases that
call for generalizations of lenses. We will discuss three such situations in
this document.

One core assumption of the lens framework is that one of the two replicas
being synchronized is ``canonical'': it stores enough information to
reconstruct the other replica in its entirety. In many cases, this is
untrue: the two replicas have some shared information, but each also has
some information that is not shared with the other.
Section~\ref{sec:symmetric_lenses} briefly discusses a symmetric variant of
lenses that relaxes this assumption; the complete dissertation will also
discuss in-depth the additional machinery needed for lens equivalence, the
algebraic structure of symmetric lenses, and a syntax for writing symmetric
lenses that includes rich support for lists and generic containers.

A second core assumption is that lenses can operate on entire replicas as a
single object. For large data sets, this can be a problem: one may not wish
to transmit entire replicas during synchronization, but instead short
descriptions of what has changed since the last synchronization point.
Section~\ref{sec:edit_lenses} briefly discusses the formalism needed to
generalize lenses so that they operate on edits; the complete dissertation
will also discuss (analogously to the discussion for symmetric lenses) edit
lens equivalence, algebraic structure, and a syntax for edit lenses that
includes some support for list and generic container operations.

Finally, the lens framework focuses itself on the problem of synchronizing
two (potentially large) replicas at a time. The main new work proposed in
this document is to produce a generalization of lenses that can synchronize
very many (potentially quite small and loosely-related) replicas. For
concreteness, imagine a multi-directional spreadsheet: each cell is a
replica. Some cells are computed from others; these computations are the
transformations that we would like to bidirectionalize.
Section~\ref{chap:spreadsheets} discusses the major goals of this work as
well as some simple solutions that we have already explored and the major
challenges we foresee.

In the remainder of the document, we will formally introduce the lens
framework (\ref{sec:background}), briefly discuss the formalisms of
symmetric and edit lenses (\ref{sec:symmetric_lenses} and
\ref{sec:edit_lenses}), introduce the hyperlens multi-directional
spreadsheet project's goals (\ref{sec:goal_statement}), challenges
(\ref{sec:design_axes}), and timeline (\ref{sec:timeline}), and give an
overview of related work (\ref{chap:related_work}).
