\documentclass{article}
\begin{document}
\begin{itemize}
    \item project proposal: bidirectional spreadsheet
        \begin{itemize}
            \item unidirectional spreadsheet: cells may have value or
                formula
            \item cells with formulas display the result of computing the
                formula
            \item when updating a cell, any formula depending on that cell
                is automatically updated
            \item to bidirectionalize: let each cell have \emph{both} value
                and formula
            \item update value independently from formula
            \item when updating a value: automatically compute new values
                for cells the formula depends on that would have that output
                as a result
        \end{itemize}
    \item why it's useful: big companies want to make predictions about
        their sales and do \emph{planning}: figuring out what local (e.g.
        store-specific or month-specific or product-specific) goals to set
        to achieve global goals
    \item TODO: examples, start with an easy one and then show what's hard
    \item why not just use a computer algebra system?
        \begin{itemize}
            \item spreadsheets are a familiar paradigm; also some research
                suggests that forming queries as a program up front (without
                being able to visualize intermediate outputs) requires
                significantly more skill than dealing with data directly
            \item want to deal with data that is not only numbers, e.g.
                product names, ID numbers, dates
            \item depending on which CAS you choose, guarantees of solution
                existence/uniqueness may be light/nonexistent -- perhaps
                some type system can help with this
            \item (pending literature search into just how exciting computer
                algebra gets) perhaps there are some operations we want --
                sorting, filtering, aggregations -- that are not readily
                available in CASs
        \end{itemize}
    \item why not just use a constraint satisfaction algorithm?
        \begin{itemize}
            \item once you've chosen an update, constraint satisfaction only
                allows information to flow one way; when there's more than
                one way to satisfy a constraint, one might like to use old
                values of the satisfied constraint to choose the new way to
                satisfy it
            \item (pending more literature search into how exciting
                constraint satisfaction gets) perhaps there is a story to
                tell about how to model the user guiding the constraint
                satisfaction process when there's many ways to satisfy a
                constraint
        \end{itemize}
    \item TODO: planned contributions
    \item TODO: also, timeline, failure modes, related work, etc.
\end{itemize}
\end{document}
