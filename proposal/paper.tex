\documentclass{article}
\begin{document}
\begin{itemize}
    \item intro: name two projects, quick overview, maybe table of contents
    \item symmetric edit lenses for trees
        \begin{itemize}
            \item what's known
                \begin{itemize}
                    \item symmetric vs. asymmetric
                    \item stateful vs. delta vs. edit
                    \item strings vs. trees
                    \item worth mentioning other stuff like quotient lenses,
                        relation lenses, dictionary lenses, matching lenses,
                        etc.? probably!
                \end{itemize}
            \item why it's worth knowing more
                \begin{itemize}
                    \item model transformation
                    \item syntax transformations like hiding code, making
                        documentation
                    \item maybe talk about machine translation, and
                        translating updated texts? a bit up in the air,
                        really, whether I want to get into that
                \end{itemize}
            \item what's hard/what I'll work on
                \begin{itemize}
                    \item defining a model of edits for trees is step one
                    \item current tree edits are mostly unconcerned with typing
                    \item really want to classify carefully when an edit
                        typechecks so that when we write lenses we know what
                        assuming a well-typed edit \emph{means}
                    \item must produce a definition of what it means to be a
                        lens: what functions are available, what laws there
                        are
                    \item syntax for creating particular lenses
                    \item some proof-of-concept indicating that the syntax
                        can generate \emph{useful} lenses -- that is, that
                        the definition of a lens is flexible enough to
                        capture the transformations we want to write and
                        that the syntax we chose is complete
                    \item may need a type system/checker for soundness
                \end{itemize}
        \end{itemize}
    \item do something good with datalog
        \begin{itemize}
            \item what's known: measure language (baroque and ad-hoc)
                \begin{itemize}
                    \item rules are ordered, and fragile to small changes
                    \item manually write ``inverses''; no checking that
                        these behave well
                    \item much preferable: automatically derive inverses
                        wherever possible with strong guarantees about
                        behavior
                \end{itemize}
            \item why it's worth knowing more: big companies want to make
                predictions about their sales and do \emph{planning}:
                figuring out what local (e.g. store-specific or
                month-specific or product-specific) goals to set to achieve
                global goals
            \item recursion and disjunction are hard, so must identify a
                fragment we can handle
            \item even the problem statement isn't totally nailed down --
                hard to find somebody who understands everything about the
                measure language and what the customers want to use it for
        \end{itemize}
    \item concrete proposal timeline (TODO: actually break things down
        enough that I know how long to assign to each of these, but this
        concrete start doesn't seem awful)
        \begin{itemize}
            \item six months on Datalog
            \item six months on trees
            \item two months writing
        \end{itemize}
    \item failure modes
        \begin{itemize}
            \item I can't understand transducers, and that turns out to be
                important
            \item I can't figure out what to do about sums
            \item Datalog is harder than I thought
            \item dialogue with LogicBlox falters (though provided we learn enough
                about what they need before this happens, can still aim at
                getting some results)
        \end{itemize}
\end{itemize}
\end{document}
