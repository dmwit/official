\documentclass{book}
\usepackage{amsmath}
\usepackage{amsthm}
\newtheorem{definition}{Definition}
\newtheorem{desiderata}{Desiderata}
\newcommand{\defined}{\downarrow}
\begin{document}

% TODO: abstract

\chapter{Introduction}
% TODO: expand each sentence here to a paragraph
Bidirectional situations are all around us. We synchronize bookmarks, keep
clones of our data on our mobile devices, edit collaboratively, use GUIs to
visualize and modify chunks of our data, etc. So designing tools that
support the creation and maintenance of these transformations is important.

Lots of nice things have been done (e.g. the asymmetric state-based lens
model and its instantiation in Boomerang), but there are many niches where
we might want more targeted support. For decentralized applications, neither
repository is canonical and you want a way to deal with that. For big data,
you want small descriptions of changes. Some data structures aren't most
naturally represented as strings, but as trees, relations, grids, graphs, or
other structures instead.

My thesis will be an exploration of these issues and what can be done to
deal with them; the major unfinished work that I will be proposing is
grid-structured data.

\chapter{Finished work}
% TODO
\section{Symmetric lenses}
% TODO
\section{Edit lenses}
% TODO

\chapter{Spreadsheets}
\documentclass{article}
\usepackage{amsmath}
\usepackage{amsthm}
\newtheorem{definition}{Definition}
\newtheorem{success}{Success criterion}
\newcommand{\defined}{\downarrow}
% TODO: success criteria: what operations are supported? (do we have a
% syntax?)
% TODO: Benjamin's proposed success criteria:
%   - generalize lenses (symmetric and asymmetric)
%   - generalize spreadsheets (x2 criteria: one is you can encode vanilla
%   spreadsheets, the other is that you can bidirectionalize a spreadsheet
%   and then when you use it in the way you would a unidirectional
%   spreadsheet its behavior is the same)
%   - give a way to modify an output and specify an input that should be the
%   only one that changes, and do it whenever that's possible
% TODO: tell the simple solutions that don't satisfy some of these solutions
%   - non-associative composition
%   - tree-structured stuff
%   - linear constraints
%   - the spreadsheet case: one method per constraint
% TODO: be clearer about design axes vs success criteria vs desiderata
\begin{document}

Lenses keep two similar pieces of data consistent; as either one evolves,
the lens finds analogous evolutions for the other. However, current lenses
don't generalize smoothly to more than two pieces of data. Spreadsheets
manage many pieces (cells) of data that are related to each other, but they
are generally unidirectional: some cells are special automatically-updated
cells, and the values in these cells are always computed by the system and
cannot be changed by the user. Constraint propagation systems generalize
spreadsheets to be many-directional when possible. However, current systems
do not use old states of the system to guide the computation of new states;
any system state which satisfies the given constraints is allowed.

The goal of the hyperlenses project is to merge the three systems, giving a
way of maintaining constraints between many pieces of data that, when given
an update to some part of the system, finds an ``analogous'' update to the
rest of the system. Below we discuss criteria on which the success of the
hyperlenses project can be judged.

In typical constraint propagation systems, there are variables and
constraints. Constraints may involve any number of variables, and are simply
relations on valuations of those variables. In the following, many of the
relations we care about will be of the form
\[\{(x_1,\ldots,x_m,y_1,\ldots,y_n) \mid f(\overline x) = g(\overline y)\}\]
and so we will simply write these as
$f(\overline x) = g(\overline y)$
when it is clear from context that a relation is expected.

\section{Simple Example}
A user might draw up a vacation expenditures spreadsheet that looks like
this:

\begin{tabular}[h]{lrrrr}
    Day     & Travel    & Lodging   & Food  & Total \\
    1       & 750       & 120       & 45    & 915   \\
    2       & 30        & 120       & 18    & 168   \\
    3       & 0         & 120       & 150   & 270   \\
    4       & 15        & 120       & 30    & 165   \\
    5       & 750       & 0         & 15    & 765   \\
    Total   & 1545      & 480       & 258   & 2283  \\
\end{tabular}

Along with the table, we would expect to see some constraints like
\begin{align*}
    \mathrm{Travel}_\mathrm{Total} &=
    \mathrm{Travel}_1+\mathrm{Travel}_2+\mathrm{Travel}_3+\mathrm{Travel}_4+\mathrm{Travel}_5
    \\
    \mathrm{Total}_1 &= \mathrm{Travel_1}+\mathrm{Lodging}_1+\mathrm{Food}_1
\end{align*}
and so on, with ten constraints in all (one each for days 1, 2, 3, 4, 5, and
Total, and one each for categories Travel, Lodging, Food, and Total). Here
are some things a user might want to do with this setup:
\begin{itemize}
    \item The user might go on another vacation, and want to make an
        estimate of how much he spent on food given his credit card balance
        at the end of the trip. To do this, he might update
        $\mathrm{Total}_\mathrm{Total}$ to his balance and look in the
        $\mathrm{Food}_\mathrm{Total}$ cell to get a guess.
    \item The user might like to plan a vacation to a certain
        location with a certain budget; then he could fix the
        $\mathrm{Total}_\mathrm{Total}$ cell and update the travel prices
        for the first and last day's plane tickets to get an estimate of how
        much he can spend on the various other days and categories while
        staying in his budget.
    \item Perhaps the user discovers that he is missing a category for
        entertainment and wants to add a new column, initially populated
        with zeros. He did not keep careful track of his daily spending for
        this on the last vacation, but he knows that in total he spent about
        \$800 on entertainment, so he updates the new
        $\mathrm{Entertainment}_\mathrm{Total}$ cell to 800.
\end{itemize}

% TODO: this organization structure doesn't quite work: section 2.2 doesn't
% fit in section 2
\section{Topology}
Fix a linearly ordered set $N$ of names and a universe $U$ of values. Since
we are dealing with partial functions, we will use the convention that
$a = b$ whenever both $a$ and $b$ are undefined or whenever $a$ and $b$ are
defined and identical. Likewise, $a \in b$ means that both $a$ and $b$ are
undefined or they are both defined and $a$ is a member of $b$.

\begin{definition}
    A \emph{valuation} is a finite map from $N$ to $U$.
\end{definition}

We generalize valuation application from names to finite sets of names using
the linear ordering on $N$: whenever $x_1 < \cdots < x_m$ is an increasing
chain, $f(\{x_1,\ldots,x_m\}) = (f(x_1),\ldots,f(x_m))$.

\begin{definition}
    A \emph{constraint} is a finite set $n \subset N$ together with a relation $R
    \subset U^{|n|}$.
\end{definition}

\begin{definition}
    A valuation $f$ \emph{satisfies} constraint $(n,R)$ when $f(n) \in R$.
\end{definition}

% \begin{definition}
%     A \emph{method} for constraint $(n,R)$ is a set $i \subset n$ and a
%     partial function $f \in U^{|n|} \to U^{|n|}$ satisfying \emph{extension}
%     and \emph{completeness}:
%     \[f(g,v) = g(v) \mbox{ whenever } v \in i\]
%     \[f(g)\defined \mbox{ whenever } \exists h \in R. h|_i = g\]
% \end{definition}
% % TODO: generalize to methods which only change some of the variables in
% % $n$?
% % TODO: define \defined; talk about coercing between tuples and valuations
% % implicitly; talk about valuation restriction

% TODO: wording: the CSG induced by a set of constraints is...
\begin{definition}
    A \emph{constraint system graph} is an undirected bipartite graph whose
    nodes are drawn from $N$ in one part and constraints in the other, and
    which has an edge $(v,(n,R))$ iff $v \in n$.
\end{definition}

One success criteria involves how exciting we allow the topology of the
constraint system graph to be.

Lenses correspond to a very strong restriction on graph topology: there are
always exactly two variable nodes with exactly one constraint connecting
them. This makes choosing an update plan particularly simple, since we must
always update the nodes attached to the single constraint.

In fact, this observation generalizes slightly: if the constraint system
graph is a tree and we allow the update of only a single node, then we may
direct the constraint graph by treating the updated node as a root and
update nodes attached to constraints in topological order (provided we
promise not to update a given node's value twice).

Here are some topologies which require greater effort.

\subsection{Cycles}
Suppose we have three variables, $x$, and $y$, and $z$, and three
constraints, one on each pair of variables. We will allow ourselves to
assume we also have a collection of methods for each individual constraint
that can take an update to one of the variables and produce a value of the
other variable that satisfies the constraint. The question now becomes: can
we take an update to one variable, say, $x$, and produce updates to the
other two that reinstate all three constraints?

The naive approach, where we compute $y$ from our assumed method that
reinstates the $\{x,y\}$ constraint and $z$ from our assumed method that
reinstates the $\{x,z\}$ constraint doesn't necessarily work, since there is
no guarantee that the $y$ and $z$ computed this way satisfy the $\{y,z\}$
constraint.

Consider our simple example above: there are two ``paths'' in the constraint
graph from the $\mathrm{Total}_\mathrm{Total}$ node to the
$\mathrm{Travel}_1$ node, namely via $\mathrm{Total}_1$ and via
$\mathrm{Travel}_\mathrm{Total}$. What we would be asking for is a guarantee
that, for example, the way we choose to spread an update over the category
totals and thereafter over the individual cells is compatible with the way
we choose to spread an update over the day totals and thereafter over the
individual cells. In the case of our simple example, we could certainly
achieve this using arithmetic facts, but in more complicated examples the
way forward is less clear.

\begin{success}
Which cyclic constraint graphs can be solved? (None? Those
that also have some other special graph structure? Those which use a
restricted syntax? Those that have some special semantic property? All?)
\end{success}

\subsection{Multiple update}
Many constraint propagation systems support the update of multiple variables
simultaneously. As discussed in our simple running example, making a
vacation plan on a budget might involve setting the grand total and the
travel costs all at once. This is distinct from setting them one at a time,
since we want the system to guarantee that all three values can coexist,
whereas when we set them one at a time each update may disrupt the values of
the other two.

\begin{success}
When can multiple variables be updated? (Never? Some statically
known sets? Dynamically known sets? Valuations with some system-specific
semantic property? Always?)
\end{success}

\subsection{Method availability}
% TODO
Not sure how to explain/motivate this one well. Currently, we demand that
all methods partition their constraint's variables into input variables and
output variables, but otherwise work with whatever methods are available.

\section{Update guidance}
Since one of the major selling points of hyperlenses over current constraint
propagation systems is that they help choose among the many possible
consistent valuations, we have:

% TODO: Benjamin thinks having this criterion at all is stupid.
\begin{success}
Can the update mechanism inspect old values while computing new values?
\end{success}

However, there are many situations where even with the old values in hand,
there are many good choices for a new consistent valuation. In these
situations, we must trade off language complexity and user flexibility.

\subsection{Intra-constraint ambiguity}
Consider the very simple constraint system which has only one constraint, $z
= x+y$. Giving a value for $z$ gives us a classical ``underconstrained
system'': there are infinitely many choices for $x$ and $y$ that satisfy
this constraint. For example, we might choose to keep $y$ and only update
$x$, we might choose to increase $x$ and $y$ by the same summand, we might
ignore the old values of $x$ and $y$ altogether and make them both be
particular fractions of $z$, we might attempt to preserve the product $x*y$,
etc. In our simple example, when we update the grand total, one reasonable
choice would be to scale all the summands by the same factor the grand total
was scaled by.

More abstractly, we might wish to have some runtime control over how
constraint solutions are being chosen in case there is ambiguity.

\begin{success}
    What control is there over the way individual constraints are resolved?
    (None -- the system makes all the choices? Spreadsheet-creation time
    control? Runtime control?)
\end{success}

\begin{success}
    How much control is there over the way individual constraints are
    resolved? (None? Some global pre-defined defaults? Constraint-level
    defaults? All -- you write a function and the system will run it?)
\end{success}

\subsection{Inter-constraint ambiguity}
Above, we discussed the possibility of constraint system graphs with cycles
in them. We observed that in such situations, it may be that no ordering of
the constraints' methods may result in a consistent state; however, there
are also situations where many orderings each result in a consistent state
-- and indeed, the chosen consistent states may even differ. As a very
simple example, consider this system that has some seemingly redundant
variables:
\begin{align*}
    z_1 &= x+y \\
    z_2 &= x+y
\end{align*}
We will assume that each constraint either allows us to update $z_i$ alone
given $x$ and $y$ or allows us to update $z_i$ and $y$ together. The first
constraint uses the update policy
\[(z_1',y') = \left(z_1 + \frac{x'-x}2, y - \frac{x'-x}2\right)\]
which spreads half the change to each variable, while the second constraint
uses the update policy
\[(z_2',y') = \left(z_2 + \frac{x'-x}3, y - \frac{2(x'-x)}3\right)\]
which spreads only a third of the change to $z_1$ and the rest to $y$.
% TODO: picture

Suppose we start from the all-zero valuation and then update $x$ to $6$.
There are (at least) two reasonable update plans that guarantee consistency:
update $z_1$ and $y$ together to $3$ and $-3$, respectively, then update
$z_2$ to $3$, or the symmetric plan that updates $z_2$ and $y$ together to
$2$ and $-4$, then updates $z_1$ to $2$.

\begin{success}
    How well are ambiguous updates handled? (Not handled -- the system picks
    one arbitrarily? With a heuristic? The programmer/user has some control?
    complete control? The system is designed in such a way that this does
    not arise?)
\end{success}

\subsection{Inter-constraint coordination}
It would be nice if the hyperlens associated with
\begin{align*}
    x &= a+b \\
    y &= x+c
\end{align*}
behaved ``similarly'' to the hyperlens associated with
\begin{align*}
    x &= b+c \\
    y &= a+x
\end{align*}
in the sense that an update to $y$ in either system resulted in the same
updates to $a$, $b$, and $c$. This is nice from a language design point of
view because it means you need not introduce separate $+$ functions for each
arity, and is nice from a usability point of view because it means that
there is no price to pay for modularity: you can split up your code into
whatever units make sense to you and get the same program out.

\begin{success}
    How much global information can constraints usefully access when making
    update choices? (None? Programmer can coordinate constraints when
    they're thrown together? Information about other constraints of the same
    form? Subgraphs with a restricted syntax can share data freely? All
    constraints understand each other?)
\end{success}

\subsection{High level guidance}
The guidance methods suggested so far are fairly low-level, and many require
an intricate understanding of the execution engine itself to employ
gainfully. One might wonder what high-level guidance can be given (and
perhaps translated by the system into low-level guidance of the kinds
suggested above) -- such as ``shunt as much of the change to this cell as
possible'', ``update as many cells as possible'', ``always go this direction
through the constraint if possible'', etc.

\begin{success}
    Can we give high-level descriptions of what a desirable update is? (No?
    Yes, but the translation to low-level instructions is undecidable?
    exponential? polynomial?)
\end{success}

\section{Miscellaneous}
There are several other criteria that are important but have no unifying
theme.

\subsection{System updates}
handling constraint system updates -- new cells, new constraints, etc.

\subsection{Efficiency}
single-pass (or at least terminating (or at least a decidable halting problem))

\subsection{Totality}
% TODO: needs significantly more flesh
Consider our simple running example and a user who tries something crazy:
updating all of the values in the Travel column (even the total). We may
have to inform the poor user that their request cannot be satisfied.

Static check: is there an instantiation of the constraints and updates to
these variables such that there is no solution?

Dynamic check: can we use facts about our constraints and this instantiation
to pick a solution, even though other constraints would make it fail?

If there is a solution, are we guaranteed to find it?

\subsection{Compositionality}
compositional (associative? commutative?)

\subsection{Complexity}
language complexity: it should not be painful to program in
\end{document}


\chapter{Related work}
% TODO
\section{Symmetric lenses}
% TODO
\section{Edit lenses}
% TODO
\section{Spreadsheets}
% TODO
\begin{itemize}
    \item why not just use a computer algebra system?
        \begin{itemize}
            \item spreadsheets are a familiar paradigm; also some research
                suggests that forming queries as a program up front (without
                being able to visualize intermediate outputs) requires
                significantly more skill than dealing with data directly
            \item want to deal with data that is not only numbers, e.g.
                product names, ID numbers, dates
            \item depending on which CAS you choose, guarantees of solution
                existence/uniqueness may be light/nonexistent -- perhaps
                some type system can help with this
            \item (pending literature search into just how exciting computer
                algebra gets) perhaps there are some operations we want --
                sorting, filtering, aggregations -- that are not readily
                available in CASs
        \end{itemize}
    \item why not just use a constraint satisfaction algorithm?
        \begin{itemize}
            \item once you've chosen an update, constraint satisfaction only
                allows information to flow one way; when there's more than
                one way to satisfy a constraint, one might like to use old
                values of the satisfied constraint to choose the new way to
                satisfy it
            \item (pending more literature search into how exciting
                constraint satisfaction gets) perhaps there is a story to
                tell about how to model the user guiding the constraint
                satisfaction process when there's many ways to satisfy a
                constraint
        \end{itemize}
\end{itemize}
\end{document}
