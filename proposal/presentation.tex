\documentclass[14pt]{beamer}
\author{Daniel Wagner}
\usepackage{tikz}
\usepackage[draft]{beamer_minimal}
\usepackage{presentation}
\usepackage{amssymb}
\usepackage{semantic}
% TODO: related work throughout
% TODO: be clear throughout about collaborators
\begin{document}

% TODO: graphic?
\title{Generalizing Lenses}
\date[University of Pennsylvania]{August 19, 2013}
\maketitle

\begin{frame}
    \frametitle{Thesis Proposal}
    \alert<2>{There are many fundamentally bidirectional settings}
    \alert<3>{that call for generalizations of traditional lenses}
    \alert<4>{where a language is possible and helpful.}
\end{frame}

% TODO: make it clear which bits predate me, which bits I helped with, which
% bits are not yet done
\begin{frame}
    \frametitle{Overview}
    \tableofcontents
\end{frame}

\section{Traditional lenses}
\begin{frame}
    \begin{center}
        \begin{tikzpicture}
            \draw[short picture tree]
                node (root) {} child {
                    node (Jan) {Jan}
                        \lolcatchildren{palindrome,gamer}
                } child {
                    node (May) {May}
                        \lolcatchildren{froghead}
                }
                ;
            \lolcatnotagsdef{froghead-fs-pic}
                {palindrome/{costume,food},gamer/{costume},froghead/{costume}}
            \fswebborder{palindrome}{froghead}
        \end{tikzpicture}
    \end{center}
\end{frame}

\begin{frame}
    \begin{center}
        \begin{tikzpicture}
            \draw[short picture tree]
                node (root) {} child {
                    node (Jan) {Jan}
                        \lolcatchildren{palindrome,gamer}
                } child {
                    node (May) {May}
                        \lolcatchildren{froghead}
                }
                ;
            \lolcatnotagsdef{froghead-fs-pic}
                {palindrome/{costume,food},gamer/{costume},froghead/{costume},burrito/{onlyface}}
            \node[fit=(burrito-web-pic)] (burrito-web-alert) {};
            \path[rounded corners,draw=alertgreen]
                ($(burrito-web-alert.north west) +(0.25,-0.25)$) rectangle
                ($(burrito-web-alert.south east) +(-0.25,0.25)$);
            \fsborder{palindrome}{froghead}
            \webborder{palindrome}{burrito}
        \end{tikzpicture}
    \end{center}
\end{frame}

\begin{frame}
    \begin{center}
        \begin{tikzpicture}
            \draw[short picture tree]
                node (root) {} child {
                    node (Jan) {Jan}
                        \lolcatchildren{palindrome,gamer}
                } child {
                    node (May) {May}
                        \lolcatchildren{froghead}
                } child {
                    node        (burrito-fs-name) {\tiny ???}
                    node[below] (burrito-fs-pic)  {\lolcat{burrito}}
                }
                ;
            \lolcatnotagsdef{burrito-fs-pic}
                {palindrome/{costume,food},gamer/{costume},froghead/{costume},burrito/{onlyface}}
            \fswebborder{palindrome}{burrito}
        \end{tikzpicture}
    \end{center}
\end{frame}

\begin{frame}
    \begin{center}
        \begin{tikzpicture}
            \draw[narrow picture tree]
                node (root) {} child {
                    node (Jan) {Jan}
                        \lolcatchildren{palindrome,gamer}
                } child {
                    node (May) {May}
                        \lolcatchildren{froghead}
                } child[sibling distance=1.33cm] {
                    node        (burrito-fs-name) {\tiny burrito.jpg}
                    node[below] (burrito-fs-pic)  {\lolcat{burrito}}
                } child[sibling distance=1.33cm] {
                    node        (withperson-fs-name) {\tiny withperson.jpg}
                    node[below] (withperson-fs-pic)  {\lolcat{withperson}}
                }
                ;
            \node[fit=(withperson-fs-name) (withperson-fs-pic)] (withperson-fs-alert) {};
            \path[rounded corners,draw=alertgreen]
                ($(withperson-fs-alert.north west) +(0.25,-0.25)$) rectangle
                ($(withperson-fs-alert.south east) +(-0.25,0.25)$);
            \lolcatnotags{(root.north -| withperson-fs-pic.east) +(1.3,0)}
                {palindrome/{costume,food},
                 gamer/{costume},
                 froghead/{costume},
                 burrito/{food,onlyface}}
            \fsborder{palindrome}{withperson}
            \webborder{palindrome}{burrito}
        \end{tikzpicture}
    \end{center}
\end{frame}

\begin{frame}
    \begin{center}
        \begin{tikzpicture}
            \draw[narrow picture tree]
                node (root) {} child {
                    node (Jan) {Jan}
                        \lolcatchildren{palindrome,gamer}
                } child {
                    node (May) {May}
                        \lolcatchildren{froghead}
                } child[sibling distance=1.33cm] {
                    node        (burrito-fs-name) {\tiny burrito.jpg}
                    node[below] (burrito-fs-pic)  {\lolcat{burrito}}
                } child[sibling distance=1.33cm] {
                    node        (withperson-fs-name) {\tiny withperson.jpg}
                    node[below] (withperson-fs-pic)  {\lolcat{withperson}}
                }
                ;
            \lolcatnotags{(root.north -| withperson-fs-pic.east) +(1.3,0)}
                {palindrome/{costume,food},
                 gamer/{costume},
                 froghead/{costume},
                 burrito/{food,onlyface},
                 withperson/{\ }}
             \fswebborder{palindrome}{withperson}
        \end{tikzpicture}
    \end{center}
\end{frame}

\begin{frame}
    % TODO: forgot create!
    \frametitle{Abstract model}
    A lens $\ell \in X \aslens Y$ has components
    \begin{align*}
        get &\in X \to Y \\
        put &\in Y \times X \to X
    \end{align*}
    \pause
    Synchronizing too often doesn't hurt.
    \begin{align*}
        get(put(y,x))&=y \\
        put(get(x),x)&=x \\
        \only<3>{\color{gray}put(y_2,put(y_1,x))&\color{gray}=put(y_2,x)}
    \end{align*}
    \only<3>{\color{gray}Not synchronizing often enough doesn't hurt.}
\end{frame}

% TODO: list the reams of work here on point-free instantiation languages,
% including for strings, relational databases, trees, etc.

\section{Symmetry}
\begin{frame}
    \begin{center}
        \begin{tikzpicture}
            \draw[narrow picture tree]
                node (root) {} child {
                    node (Jan) {Jan}
                        \lolcatchildren{palindrome,gamer}
                } child {
                    node (May) {May}
                        \lolcatchildren{froghead}
                } child[sibling distance=1.33cm] {
                    node        (burrito-fs-name) {\tiny burrito.jpg}
                    node[below] (burrito-fs-pic)  {\lolcat{burrito}}
                } child[sibling distance=1.33cm] {
                    node        (withperson-fs-name) {\tiny withperson.jpg}
                    node[below] (withperson-fs-pic)  {\lolcat{withperson}}
                }
                ;
            \lolcatnotags{(root.north -| withperson-fs-pic.east) +(1.3,0)}
                {palindrome/{costume,food},
                 gamer/{costume},
                 froghead/{costume},
                 burrito/{food,onlyface},
                 withperson/{\ }}
             \fswebborder{palindrome}{withperson}
        \end{tikzpicture}
    \end{center}
\end{frame}

\begin{frame}
    \begin{center}
        \begin{tikzpicture}
            \draw[narrow picture tree]
                node (root) {} child {
                    node (Jan) {Jan}
                        \lolcatchildren{palindrome,gamer}
                } child {
                    node (May) {May}
                        \lolcatchildren{froghead}
                } child[sibling distance=1.33cm] {
                    node        (burrito-fs-name) {\tiny burrito.jpg}
                    node[below] (burrito-fs-pic)  {\lolcat{burrito}}
                } child[sibling distance=1.33cm] {
                    node        (withperson-fs-name) {\tiny withperson.jpg}
                    node[below] (withperson-fs-pic)  {\lolcat{withperson}}
                }
                ;
            \lolcattags{(root.north -| withperson-fs-pic.east) +(1.3,0)}
                {palindrome/{costume,food},
                 gamer/{costume},
                 froghead/{costume},
                 burrito/{food,onlyface},
                 withperson/{\ }}
             \fswebborder{palindrome}{withperson}
        \end{tikzpicture}
    \end{center}
\end{frame}

\begin{frame}
    \begin{center}
        \begin{tikzpicture}
            \draw[narrow picture tree]
                node (root) {} child {
                    node (Jan) {Jan} child {
                        node        (palindrome-fs-name) {\tiny palindrome.jpg}
                        node[below] (palindrome-fs-tag)  {\tiny [costume,food]} (palindrome-fs-tag)
                        node[below] (palindrome-fs-pic)  {\lolcat{palindrome}}
                    } child {
                        node        (gamer-fs-name) {\tiny gamer.jpg}
                        node[below] (gamer-fs-tag)  {\tiny [costume]} (gamer-fs-tag)
                        node[below] (gamer-fs-pic)  {\lolcat{gamer}}
                    }
                } child {
                    node (May) {May} child {
                        node        (froghead-fs-name) {\tiny froghead.jpg}
                        node[below] (froghead-fs-tag)  {\tiny [costume]} (froghead-fs-tag)
                        node[below] (froghead-fs-pic)  {\lolcat{froghead}}
                    }
                } child[sibling distance=1.33cm] {
                    node        (burrito-fs-name) {\tiny burrito.jpg}
                    node[below] (burrito-fs-tag)  {\tiny [food,onlyface]} (burrito-fs-tag)
                    node[below] (burrito-fs-pic)  {\lolcat{burrito}}
                } child[sibling distance=1.33cm] {
                    node        (withperson-fs-name) {\tiny withperson.jpg}
                    node[below] (withperson-fs-tag)  {\tiny [\ ]} (withperson-fs-tag)
                    node[below] (withperson-fs-pic)  {\lolcat{withperson}}
                }
                ;
            \lolcattags{(root.north -| withperson-fs-pic.east) +(1.3,0)}
                {palindrome/{costume,food},
                 gamer/{costume},
                 froghead/{costume},
                 burrito/{food,onlyface},
                 withperson/{\ }}
             \fswebborder{palindrome}{withperson}
             \path[use as bounding box] (fs.north west) -- (web.south east);
             \draw<2>[line width=0.6cm,red!70!black] (fs.center)
                circle (3.2cm)
                +(135:3.2cm) -- +(315:3.2cm)
                ;
        \end{tikzpicture}
    \end{center}
\end{frame}

\begin{frame}
    \begin{center}
        \begin{tikzpicture}
            \draw[narrow picture tree]
                node (root) {} child {
                    node (Jan) {Jan}
                        \lolcatchildren{palindrome,gamer}
                } child {
                    node (May) {May}
                        \lolcatchildren{froghead}
                } child[sibling distance=1.33cm] {
                    node        (burrito-fs-name) {\tiny burrito.jpg}
                    node[below] (burrito-fs-pic)  {\lolcat{burrito}}
                } child[sibling distance=1.33cm] {
                    node        (withperson-fs-name) {\tiny withperson.jpg}
                    node[below] (withperson-fs-pic)  {\lolcat{withperson}}
                }
                ;
            \lolcattags{(root.north -| withperson-fs-pic.east) +(1.3,0)}
                {palindrome/{costume,food},
                 gamer/{costume},
                 froghead/{costume},
                 burrito/{food,onlyface},
                 withperson/{\ }}
             \fswebborder{palindrome}{withperson}
             \path[tiny complement tree] (fs.south) ++(-0.8cm, -0.6cm)
                node (complement-root) {} child {
                    node (complement-Jan) {Jan} child {
                        node (complement-palindrome) {palindrome\strut}
                    } child {
                        node (complement-gamer) {gamer\strut}
                    }
                } child {
                    node (complement-May) {May} child {
                        node (complement-froghead) {froghead\strut}
                    }
                } child {
                    node (complement-burrito) {burrito\strut}
                } child {
                    node (complement-withperson) {withperson\strut}
                }
                ;
                \begin{scope}[start chain=going below,node distance=0,inner sep=0.2ex,font=\tiny]
                    \path (complement-root) ++(3.5cm,-0.1cm)
                        node[on chain] (complement-palindrome-tag) {[costume,food]}
                        node[on chain] (complement-gamer-tag)      {[costume]}
                        node[on chain] (complement-froghead-tag)   {[costume]}
                        node[on chain] (complement-burrito-tag)    {[food,onlyface]}
                        node[on chain] (complement-withperson-tag) {[\ ]}
                        ;
                \end{scope}
                \path<2>
                    node[fit=(complement-root)
                             (complement-palindrome)
                             (complement-palindrome-tag)
                             (complement-withperson-tag)] (complement) {}
                    (complement) node[anchor=center,scale=3] (complement-check)
                        {\Large\color{alertgreen}\checkmark}
                    ;
        \end{tikzpicture}
    \end{center}
\end{frame}

\begin{frame}
    \frametitle{Abstract model}
    A lens $\ell \in X \sslens Y$ has a set $C$ and components
    \begin{align*}
        putr &\in X\times C \to Y\times C \\
        putl &\in Y\times C \to X\times C
    \end{align*}
    \pause
    Synchronizing too often doesn't hurt.
    % TODO: why isn't this centered?
    \begin{center}
        \inference{putr(x,c)=(y,c')}{putl(y,c')=(x,c')}
    \end{center}
    % TODO:
    %\only<3>{\color{gray}Not synchronizing often enough doesn't hurt.}
\end{frame}

\section{Edits}
\section{Multidirectionality}
\section{Logistics}
\end{document}
