% TODO: comments from Benjamin:
% A small suggestion is that you may want to explain stratification.
%
% A slightly bigger one is that if Tiresias is very close to your work you
% may want to explain it in more detail.
%
% An open ended one is that you may get questions at your defense about how
% sure you are that you've found all the work that's related.  But you can
% do some more literature search between now and then -- this seems fine for
% the document.

\section{Symmetric lenses}
% TODO: cleanup (remove ifs, rework wording so it fits in this document)
\newif \iftext  \texttrue
\newif \iffull  \fulltrue
\newif \ifdraft \draftfalse
\newif \ifdelta \deltafalse
\newif \iflater \laterfalse  % (for things that we're going to think about later)

% If you want to build without tikz, put
%    \tikzfalse
% in a separate texdirectives.tex file...
\newif \iftikz  \tikztrue
There is a large literature on lenses and related approaches to
propagating updates between connected structures.  We discuss only the most
closely related work here; good general surveys of the area can be found
in~\cite{FosterThesis,DBLP:conf/icmt/CzarneckiFHLST09}.  Connections to the
literature on {\em view update} in databases are surveyed
in~\cite{Focal2005-shortcite}. \iffull A short version of this paper is available
in~\cite{HofmannPierceWagner10}.\fi

The first symmetric approach to update propagation was proposed by
Meertens~\cite{Meertens98} and followed up \iffull in the context of
model-driven design \fi by Stevens~\cite{Stevens07},
Diskin~\cite{DBLP:conf/models/Diskin08}, and Xiong, et
al~\cite{xiong2009supporting}.
%
Meertens suggests modeling synchronization between two sets
$X$ and $Y$ by a {\em consistency relation} $R\subseteq
X\times Y$ and two {\em consistency maintainers}
$\triangleleft: X\times Y\rightarrow X$ and $\triangleright: X\times Y
\rightarrow Y$ such that $(x\triangleleft y) \relR y$ and
$x \relR (x\triangleright y)$ always hold, and such that $x \relR y$ implies
$x \triangleleft y = x$ and $x \triangleright y = y$.

The main advantage of symmetric lenses over consistency maintainers is
their closure under composition. Indeed, all of the aforementioned
authors note that, in general, consistency maintainers do not compose
and view this as a drawback.
%
Suppose that we have relations $R\subseteq X\times Y$ and
$R'\subseteq Y\times Z$ maintained by $\triangleright,\triangleleft$
and $\triangleright', \triangleleft'$, resp. If we want to construct a
maintainer for the composition $R;R'$, we face the problem that, given
$x\in X$ and $z\in Z$, there is no canonical way of coming up with a
$y\in Y$ that will allow us to use either of the existing maintainer
functions. Concretely, Meertens gives the following counterexample.
Let $X$ be the set of nonempty context free grammars over some alphabet, and let
$Y$ be the set of words over that same alphabet. Let $R\subseteq
X\times Y$ be given by $G \relR x\iff x\in L(G)$. It is easy to define
computable maintainer functions making this relation a constraint
maintainer. Composing this relation with its opposite yields an
undecidable relation (namely, whether the intersection of two context-free
grammars is nonempty), so there cannot be computable maintainer functions.

We can transform any constraint maintainer into a  symmetric lens as
follows: take the relation $R$ itself (viewed as a set of pairs) as
the complement, and define $\putl(x',(x,y))=(x'\triangleright
y,(x',x'\triangleright y))$ and similarly
for $\putr$. If we compose such a symmetric lens with its opposite
we obtain $R\times R\op$ as the complement and, for example,
$\putr(x',((x_1,y_1),(y_2,x_2))) =
(x_2\triangleleft(x'\triangleright y_1), ((x',x'\triangleright
y_1),(x'\triangleright y_1,x_2\triangleleft(x'\triangleright y_1))))$.
%
For Meertens' counterexample, we would have complements of the form
$((G_1,w_1),(w_2,G_2))$, with $w_1\in L(G_1)$
and $w_2\in L(G_2)$; ``$\putr$''-ing a new grammar
$G_1'$ through the composed lens yields the complement
$((G_1',w_1'),(w_1',G_2'))$, where $w_1'$ is $w_1$ if $w_1\in L(G_1)$ and
some default otherwise, and where $G_2'=G_2$ if $w_1'\in L(G_2)$ and
$S{\rightarrow}w_1'$ (where $S$ is the start state) otherwise. We observe
that there is a property of lenses analogous to Meertens' requirement that
$x \relR y$ implies $x \triangleleft y = x$. This property is not
necessarily preserved by composition, and in particular the lens described
above for synchronizing languages does not have it.
%
Meertens recommends using a {\em chain} of consistency maintainers in such a
situation to achieve a similar effect; however, the properties of such
chains have not been explored.

%%%%%%%%%%%%%%%%%%%%%%%%%%%%%%%%%

For asymmetric lenses, a number of alternative \iffull choices of behavioral
\fi
laws
have been explored.  Some of these are \iffull strictly \fi weaker than ours; for
example, a number of papers from a community of researchers based in Tokyo
replace the \rn{PutGet} law with a somewhat looser \rn{PutGetPut} law,
permitting a broader range of useful behaviors for lenses that duplicate
information.  It would be interesting to see what kind of categorical
structures arise from these choices.  The proposal by Matsuda et
al.~\cite{matsuda2007btb} is particularly interesting because it also employs
the idea of complements.  Conversely, stronger laws can be imagined, such as
the \rn{PutPut} law discussed by Foster et
al.~\cite{Focal2005-shortcite}\iffull{} and the more refined variants
in~\cite{updatable-security-views}\fi\iflater\finish{Say something more
  about this?}\fi.

A different foundation for defining lenses by recursion was explored by
Foster et al.~\cite{Focal2005-shortcite}, using standard tools from domain
theory to define monotonicity and continuity for lens combinators
parametrized on other lenses.  The main drawback of this approach is that
the required (manual) proofs that such recursive lenses are total tend to be
somewhat intricate.  By contrast, we expect that our initial-algebra
approach can be equipped with automatic proofs of totality (that is, choices
of the weight function $w$) in many cases of interest.

\ifdelta
\finish{
Updates: (note that the point in our paper is not update as such but rather
how to marry it with lenses)  \finish{BCP will look for a good canonical
  reference for updates}
\begin{itemize}
\item lens-like things that have used deltas~\cite{Hu04,Meertens98,MuAlgebraic2004,HuModels07}
\item old update transformation
\item old view update stuff
\item new Waterloo stuff
\begin{itemize}
\item flexibility, not efficiency
\item no proposal for syntax---composition is the only operator they study
at all
\end{itemize}
\end{itemize}
}
\fi
\section{Edit lenses}
% TODO: cleanup (remove ifs, rework wording so it fits in this document)
\newif \iffull  \fulltrue
\newif \ifspaceisnoissue  \spaceisnoissuefalse
\newif \ifdraft \drafttrue
\newif \ifanon  \anonfalse
\newif \iflater \laterfalse  % (for things that we're going to think about later)
\newif \iffailed \failedfalse % to remove some bits we think we don't need

%\fullfalse \draftfalse  % for final version

\iffull\spaceisnoissuetrue\fi  % space is no issue in the full version

% If you want to build without tikz, put
%    \tikzfalse
% in a separate texdirectives.tex file...
\newif \iftikz  \tikztrue
The most closely related attempt at developing a theory of update
propagation is \cite{Diskin-Delta11} by Diskin et al. Their starting
point is the observation (also discussed in \cite{Matching10}) that discovery of
edits should be decoupled from their propagation. They thus propose a
formalism, \emph{sd-lenses}, for the propagation of edits across
synchronized data structures, bearing some similarities with our
edit lenses. The replicas, which we model as modules, are there modeled
as categories (presented as reflexive graphs).
Thus, for any two states $x,x'$ there is a set of
edits $X(x,x')$. An sd-lens then comprises two reflexive graphs $X,Y$
and for any $x\in X$ and $y\in Y$ a set $C(x,y)$ of
``correspondences'' which roughly correspond to our
complements. Forward and backward operations similar to our $\dputl$ and
$\dputr$ then complete the picture. No concrete
examples are given of sd-lenses, no composition, no notion of equivalence, and
no combinators for constructing sd-lenses; the focus of the paper is
rather on the discovery of suitable axioms, such as invertibility and
undoability of edits, and a
generalization of {\em hippocraticness} in the sense of
Stevens~\cite{Stevens07}. They also develop a comparison
with the state-based framework. In
our opinion, the separation of edits and correspondences according to
the states that they apply to or relate has two important
disadvantages.  First, in our examples, it is often the case that one
and the same edit applies to more than one state and can be
meaningfully propagated (and more compactly represented) as such. For example, while many of the
container edits tend to only work for a particular shape, they are
completely polymorphic in the contents of the container. Second, the
fact that state sets are already categories suggests
that a category of sd-lenses would be
2-categorical in flavor, entailing extra technical difficulties such as
coherence conditions.

%% \finish{Here's a quick start from a brief scan of their most recent
%%   paper~\cite{Diskin-Delta11} by BCP...
%%   \begin{itemize}
%%   \item Their motivations and goals are exactly the same.
%%   \item The technicalities of their approach are pretty dense.  I haven't
%%   internalized them yet.  In particular, I don't have a good intuition for
%%   their ``sameness'' relations---what they say makes sense for ``flat''
%%   structures like lists or simple graphs, but not for more structured data
%%   (where you'd want to know about correspondences at various levels of
%%   structure).
%%   \item They don't say anything about the size of their deltas, and at
%%   least a naive representation would be big.  We're much more careful about
%%   this.
%%   \item They propose two new laws (weak invertability and undoability).  I'm
%%   not sure what to say about these, but I guess it's an important point of
%%   comparison, since it's one of the main points of their paper.
%%   \item They don't handle $\missing$ (a small point)
%%   \item They don't define any combinators, just the semantic space itself (a
%%   larger point, and related)
%%   \item Their Definition 19 and Theorem 5 relate their delta-lenses to our
%%   symmetric lenses.  I'm not completely sure how to interpret it (does our
%%   ``trivial module'' correspond to their ``simple graph''?), but in any case
%%   our result is stronger because it goes both directions.  (Their
%%   characterization of our symmetric lenses is a little bit wrong---it puts
%%   $C$ in the wrong place---but I'm not sure this matters.)
%%   \end{itemize}
%% }

Meertens's seminal paper on {\em constraint maintainers}~\cite{Meertens98}
discusses a form of containers for lists equipped with a notion of edits
similar to our edit language for lists, but does not develop a general
theory of edit-transforming constraint maintainers.

A long series of papers from the group at the University of Tokyo
\cite[etc.]{Hu04, Mu2004, MuAlgebraic2004, HuModels07,
  Hidaka10}\iffull\discuss{double-check these, and add more, and add to abstract
  too}\fi{} deal with the alignment issue using an approach that might be
characterized as a hybrid of state-based and edit-based.  Lenses work with
whole states, but these states are internally annotated with tags showing
where edits have been applied---e.g., marking inserted or deleted elements
of lists.
%
Barbosa et al.'s {\em matching lenses}~\cite{Matching10} offer another
approach to dealing with issues of alignment in the framework of pure
state-based lenses.


%% \noindent
%% Other things we definitely need to compare to:
%% \begin{itemize}
%% \item Diskin's ``tile algebras'' \cite{DBLP:conf/gttse/Diskin09}
%% \item The other delta-lens papers that Diskin refers to in the intro
%% of~\cite{Diskin-Delta11}
%% \item Other papers addressing the alignment problem in different ways (e.g.,
%% our Matching Lenses~\cite{Matching10}, Tokyo group papers such
%% as~\cite{Mu2004} and maybe the ICFP10 paper on bidirectional graph
%% transformations)
%% \item Maybe some papers on edits in the context of version management (see
%% last year's grant proposal for some citations)
%% \item other lens-like things that have used some kind of deltas~\cite{HuModels07}
%% \begin{itemize}
%% \item \finish{Meertens \cite{Meertens98} introduces edit operations in section 5.3
%% (p. 68ff) to talk about edits to lists.  I have not grokked yet exactly how
%% all this works, or how it fits into his general framework of constraint
%% maintainers.  Some possible text:}

%% \item
%% \end{itemize}
%% \end{itemize}

%% \iflater
%% \finish{
%% Other things to think about:
%% \begin{itemize}
%%     \item other research on edit lenses
%%     \item practical tools that use notions of an edit
%%     \item Is there any correspondence with other tools like SVN, Unison,
%%     ...?
%% \item operation transform papers?
%% \end{itemize}
%% }
%% \fi

\section{Spreadsheets}
The spreadsheet model proposed here draws significant inspiration from the
field of constraint programming systems, which dates back to at least the
Sketchpad drawing system of 1963~\cite{sutherland1964sketch}. A good survey
of the huge body of work done in this area is given
in~\cite{wallace1996practical}.
%
In constraint programming languages, programs typically include a series of
declarations defining what valuations are valid and invalid; the language
runtime is then tasked with finding a valuation which satisfies the
constraints. Our proposed work would maintain this broad framework, but
extend it by providing more control over which of many possible valuations
may be chosen. In particular, our proposal is to model information both
about satisfying valuations (as is done in previous systems) \emph{and}
about the evolution of valuations. To achieve this, the modules responsible
for re-instating individual declared constraints must be given data about
both the old satisfying valuation and the desired new valuation.
Additionally, we propose an investigation of methods for separately
specifying whether a valuation is valid and how desirable a valuation is.

There is a chain of work on DeltaBlue, a particular constraint programming
system, which adds the ability to rank constraints, and break low-ranked
constraints during the constraint solution
stage~\cite{sannella1994analyzing,sannella1994skyblue,sannella1993multi}.
This gives one way of influencing ambiguous solutions: add low-ranked
constraints expressing the desirable properties of your valuation. When
there are multiple valuations possible on the high-ranked constraints, the
low-ranked ones may be used to choose between them. We believe some
properties of ``desirability'' are not naturally representable as
constraints, especially when ``desirable'' means ``the new valuation is
close to the old one in this way''.

% TODO: also, many (most?) of the constraint propagation systems are about
% making GUIs, which we aren't really that interested in primarily: the data
% we're munging is user-visible (perhaps this is a serious point to make!)

The algebraic properties of constraint propagation systems have been
explored somewhat~\cite{jarvi2012expressing}. J\"arvi et al. discuss the
model we intend to used as a starting point, and show that it can be
decomposed into a structure with an associative, commutative composition and
an algorithm which traverses this structure for update planning without
losing efficiency. However, because determinism is not a goal for them, they
make no efforts to resolve the ambiguity that can arise from the existence
of multiple update plans.

There are several systems with proposed solutions that are not based on
constraint propagation systems. For example, a GUI resembling a spreadsheet
is discussed in~\cite{stadelmann1993spreadsheet}. Numerical constraints --
including constraints representing assignments of values to particular cells
-- are shipped out to Mathematica for solution. A significant advantage of
this approach is that it can take advantage of the significant solving power
of Mathematica. However, this methodology is restricted to data types known
by Mathematica; does not provide old cell values to use when updating the
spreadsheet; provides little control over which of many satisfying
valuations are chosen; and does not attempt to make any guarantees about
totality.

Another such system is Tiresias, which extends Datalog with bidirectional
capabilities for numeric computations~\cite{meliou2012tiresias}. They give a
variant of Datalog that allows for nondeterministic predicates (that is,
where tuples may or may not appear) and show how to pick a deterministic
instantiation of those predicates that satisfies a Datalog query. The choice
of instantiation can also be guided by an objective function to be maximized
or minimized.  This seems to be a very promising approach, but does have a
few important limitations: first, there is a topological constraint (the
Datalog query must be stratified); and second, it is not clear that the
approach can be easily generalized beyond the small collection of arithmetic
predicates that it currently supports.

% Probably not worth discussing:
% Caravan (summarize the database so that I can change my query and still get a fast answer just by looking at the summary)
% Automatic Generation and Maintenance of Correct Spreadsheets (takes 2D structure seriously, but is unidirectional so not all that related)
% Towards the Bidirectionalization of Spreadsheet Formulas (no actual results, just a claim that it's worth studying)

% TODO: see spreadsheet_resources.txt in triple-threat repository
