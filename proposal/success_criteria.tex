Lenses keep two similar pieces of data consistent; as either one evolves,
the lens finds analogous evolutions for the other. However, current lenses
don't generalize smoothly to more than two pieces of data. Spreadsheets
manage many pieces (cells) of data that are related to each other, but they
are generally unidirectional: some cells are special automatically-updated
cells, and the values in these cells are always computed by the system and
cannot be changed by the user. Constraint propagation systems generalize
spreadsheets to be many-directional when possible. However, current systems
do not use old states of the system to guide the computation of new states;
any system state which satisfies the given constraints is allowed.

The goal of the hyperlenses project is to merge the three systems, giving a
way of maintaining constraints between many pieces of data that, when given
an update to some part of the system, finds an ``analogous'' update to the
rest of the system. Below we discuss criteria on which the success of the
hyperlenses project can be judged.

In typical constraint propagation systems, there are variables and
constraints. Constraints may involve any number of variables, and are simply
relations on valuations of those variables. In the following, many of the
relations we care about will be of the form
\[\{(x_1,\ldots,x_m,y_1,\ldots,y_n) \mid f(\overline x) = g(\overline y)\}\]
and so we will simply write these as
$f(\overline x) = g(\overline y)$
when it is clear from context that a relation is expected.

\section{Simple example}
A user might draw up a vacation expenditures spreadsheet that looks like
this:

\begin{tabular}[h]{lrrrr}
    Day     & Travel    & Lodging   & Food  & Total \\
    1       & 750       & 120       & 45    & 915   \\
    2       & 30        & 120       & 18    & 168   \\
    3       & 0         & 120       & 150   & 270   \\
    4       & 15        & 120       & 30    & 165   \\
    5       & 750       & 0         & 15    & 765   \\
    Total   & 1545      & 480       & 258   & 2283  \\
\end{tabular}

Along with the table, we would expect to see some constraints like
\begin{align*}
    \mathrm{Travel}_\mathrm{Total} &=
    \mathrm{Travel}_1+\mathrm{Travel}_2+\mathrm{Travel}_3+\mathrm{Travel}_4+\mathrm{Travel}_5
    \\
    \mathrm{Total}_1 &= \mathrm{Travel_1}+\mathrm{Lodging}_1+\mathrm{Food}_1
\end{align*}
and so on, with ten constraints in all (one each for days 1, 2, 3, 4, 5, and
Total, and one each for categories Travel, Lodging, Food, and Total). Here
are some things a user might want to do with this setup:
\begin{itemize}
    \item The user might go on another vacation, and want to make an
        estimate of how much he spent on food given his credit card balance
        at the end of the trip. To do this, he might update
        $\mathrm{Total}_\mathrm{Total}$ to his balance and look in the
        $\mathrm{Food}_\mathrm{Total}$ cell to get a guess.
    \item The user might like to plan a vacation to a certain
        location with a certain budget; then he could fix the
        $\mathrm{Total}_\mathrm{Total}$ cell and update the travel prices
        for the first and last day's plane tickets to get an estimate of how
        much he can spend on the various other days and categories while
        staying in his budget.
    \item Perhaps the user discovers that he is missing a category for
        entertainment and wants to add a new column, initially populated
        with zeros. He did not keep careful track of his daily spending for
        this on the last vacation, but he knows that in total he spent about
        \$800 on entertainment, so he updates the new
        $\mathrm{Entertainment}_\mathrm{Total}$ cell to 800.
\end{itemize}

\section{Goal statement}
{\bf A hyperlens should be a generalization of both lenses and spreadsheets that
supports high-level planning.}

Hyperlenses should generalize lenses. It should be true that there is a
behavior-preserving embedding of asymmetric, state-based lenses: that is, we
can identify two variables in the hyperlens system whose values correspond
to states of the asymmetric lens' two repositories, and the values in the
hyperlens system evolve in the same way they would evolve when running the
lens itself. Moreover, we demand that there be a ``hyperlens composition''
that preserves this property: the embedding of a composition of lenses is
the composition of their embeddings.

A stretch goal is to generalize symmetric, state-based lenses in a similar
way, and again to find a composition operator (potentially different from
the previous one) that corresponds to symmetric lens composition.

Hyperlenses should generalize spreadsheets. It is unreasonable to demand
that the hyperlens framework be capable of bidirectionalizing all
spreadsheets in a reasonable way. However, on the class of spreadsheets that
can be bidirectionalized, it should be the case that using the corresponding
hyperlens as if it were a unidirectional spreadsheet produces the same
answers as the original spreadsheet would.

A stretch goal is to generalize spreadsheets in the sense that any
spreadsheet can be expressed as a (possibly still unidirectional) hyperlens
with the same behavior.

Hyperlenses should support high-level planning. As with all bidirectional
systems, there will be some updates which can be spread through the
remainder of the system in many ways. Support for high-level planning means
that there is some holistic language (that is, which does not require
intimate knowledge of the structure of the term used to define the
hyperlens) for expressing the relative desirability of the various coherent
updates to the system. The system should then be able to compute the most
desirable update. For example, one such language might be, ``spread as much
of the change as possible to such-and-such a variable''.

% TODO: cite a few use cases with the high-level planning we want for each
\section{Simplistic solutions}
Some particularly simple regimes have already been explored. Four such
regimes are discussed below, but it will be helpful to have a few
conventions in place first.

Fix a linearly ordered set $N$ of names and a universe $U$ of values. Since
we are dealing with partial functions, we will use the convention that
$a = b$ whenever both $a$ and $b$ are undefined or whenever $a$ and $b$ are
defined and identical. Likewise, $a \in b$ means that both $a$ and $b$ are
undefined or they are both defined and $a$ is a member of $b$.

\begin{definition}
    A \emph{valuation} is a finite map from $N$ to $U$.
\end{definition}

We generalize valuation application from names to finite sets of names using
the linear ordering on $N$: whenever $x_1 < \cdots < x_m$ is an increasing
chain, $f(\{x_1,\ldots,x_m\}) = (f(x_1),\ldots,f(x_m))$.

\begin{definition}
    A \emph{constraint} is a finite set $n \subset N$ together with a relation $R
    \subset U^{|n|}$.
\end{definition}

\begin{definition}
    A valuation $f$ \emph{satisfies} constraint $(n,R)$ when $f(n) \in R$.
\end{definition}

\begin{definition}
    The \emph{constraint system graph} induced by a set of constraints is
    the undirected bipartite graph whose nodes are drawn from $N$ in one
    part and constraints in the other, and which has an edge $(v,(n,R))$ iff
    $v \in n$.
\end{definition}

\subsection{Linear constraints}
% TODO

\subsection{Spreadsheets}
% TODO: write about unidirectional methods

\subsection{Tree topology}
% TODO

Lenses correspond to a very strong restriction on graph topology: there are
always exactly two variable nodes with exactly one constraint connecting
them. This makes choosing an update plan particularly simple, since we must
always update the nodes attached to the single constraint.

In fact, this observation generalizes slightly: if the constraint system
graph is a tree and we allow the update of only a single node, then we may
direct the constraint graph by treating the updated node as a root and
update nodes attached to constraints in topological order (provided we
promise not to update a given node's value twice).

\subsection{Non-associative composition}
% TODO: pick a better subsection name
% TODO: write

\section{Design axes}

A full solution could reasonably build on either the restrictive constraints
or the restrictive topology solutions given above. Because it seems
difficult to extend the restrictive constraints solution sufficiently to
achieve our top-level goal of embedding all asymmetric, state-based lenses,
the approach of extending the restrictive topology solution seems more
promising. Below, we discuss some of the difficulties that should be addressed
by a successful extension.

There is a distinction between the dynamic and static semantics of a
constraint system graph. Unless specified otherwise, all discussion is of
the static semantics.
\begin{definition} Semantics:
    \begin{description}
        \item[Dynamically ambiguous] means the current set of constraints and
            requested updates have multiple satisfying valuations.
        \item[Dynamically unsolvable] means the current set of constraints and
            requested updates have no satisfying valuations.
        \item[Statically ambiguous] means there is a set of requested updates
            and a set of constraints whose graph is the current one which is
            dynamically ambiguous.
        \item[Statically unsolvable] means there is a set of requested updates
            and a set of constraints whose graph the current one which is
            dynamically unsolvable.
    \end{description}
\end{definition}

\subsection{Sources of ambiguity}
\subsubsection{Intra-constraint ambiguity}
Consider the very simple constraint system which has only one constraint, $z
= x+y$. Giving a value for $z$ gives us a classical ``underconstrained
system'': there are infinitely many choices for $x$ and $y$ that satisfy
this constraint. For example, we might choose to keep $y$ and only update
$x$, we might choose to increase $x$ and $y$ by the same summand, we might
ignore the old values of $x$ and $y$ altogether and make them both be
particular fractions of $z$, we might attempt to preserve the product $x*y$,
etc. In our simple example, when we update the grand total, one reasonable
choice would be to scale all the summands by the same factor the grand total
was scaled by.

More abstractly, we might wish to have some runtime control over how
constraint solutions are being chosen in case there is ambiguity.

\begin{desiderata}
    Have programmer-level control over the resolution of individual
    constraints.
\end{desiderata}
\begin{desiderata}
    Have high-level control over the resolution of individual constraints.
\end{desiderata}

\subsubsection{Cycles and inter-constraint ambiguity}
Above, we discussed the possibility of constraint system graphs with cycles
in them. We observed that in such situations, it may be that no ordering of
the constraints' methods may result in a consistent state; however, there
are also situations where many orderings each result in a consistent state
-- and indeed, the chosen consistent states may even differ. As a very
simple example, consider this system that has some seemingly redundant
variables:
\begin{align*}
    z_1 &= x+y \\
    z_2 &= x+y
\end{align*}
We will assume that each constraint either allows us to update $z_i$ alone
given $x$ and $y$ or allows us to update $z_i$ and $y$ together. The first
constraint uses the update policy
\[(z_1',y') = \left(z_1 + \frac{x'-x}2, y - \frac{x'-x}2\right)\]
which spreads half the change to each variable, while the second constraint
uses the update policy
\[(z_2',y') = \left(z_2 + \frac{x'-x}3, y - \frac{2(x'-x)}3\right)\]
which spreads only a third of the change to $z_1$ and the rest to $y$.
% TODO: picture

Suppose we start from the all-zero valuation and then update $x$ to $6$.
There are (at least) two reasonable update plans that guarantee consistency:
update $z_1$ and $y$ together to $3$ and $-3$, respectively, then update
$z_2$ to $3$, or the symmetric plan that updates $z_2$ and $y$ together to
$2$ and $-4$, then updates $z_1$ to $2$.

\begin{desiderata}
    Provide high-level control over ambiguous cycles.
\end{desiderata}

\subsection{Sources of insolubility}
\subsubsection{Cycles}
Suppose we have three variables, $x$, and $y$, and $z$, and three
constraints, one on each pair of variables. We will allow ourselves to
assume we also have a collection of methods for each individual constraint
that can take an update to one of the variables and produce a value of the
other variable that satisfies the constraint. The question now becomes: can
we take an update to one variable, say, $x$, and produce updates to the
other two that reinstate all three constraints?

The naive approach, where we compute $y$ from our assumed method that
reinstates the $\{x,y\}$ constraint and $z$ from our assumed method that
reinstates the $\{x,z\}$ constraint doesn't necessarily work, since there is
no guarantee that the $y$ and $z$ computed this way satisfy the $\{y,z\}$
constraint.

Consider our simple example above: there are two ``paths'' in the constraint
graph from the $\mathrm{Total}_\mathrm{Total}$ node to the
$\mathrm{Travel}_1$ node, namely via $\mathrm{Total}_1$ and via
$\mathrm{Travel}_\mathrm{Total}$. What we would be asking for is a guarantee
that, for example, the way we choose to spread an update over the category
totals and thereafter over the individual cells is compatible with the way
we choose to spread an update over the day totals and thereafter over the
individual cells. In the case of our simple example, we could certainly
achieve this using arithmetic facts, but in more complicated examples the
way forward is less clear.

\begin{desiderata}
    Handle dynamically solvable cycles.
\end{desiderata}

\subsubsection{Multiple update}
Many constraint propagation systems support the update of multiple variables
simultaneously. As discussed in our simple running example, making a
vacation plan on a budget might involve setting the grand total and the
travel costs all at once. This is distinct from setting them one at a time,
since we want the system to guarantee that all three values can coexist,
whereas when we set them one at a time each update may disrupt the values of
the other two.

\begin{desiderata}
    Identify solvable multiple updates.
\end{desiderata}

\subsection{Other difficulties}

% TODO: motivate why each of these is nice and why each is hard
\begin{desiderata}
    Provide an associative, commutative composition operation.
\end{desiderata}
\begin{desiderata}
    Reinstate coherence in a single pass: for each constraint, execute one
    method at most once.
\end{desiderata}
\begin{desiderata}
    Provide a syntax for basic spreadsheet programming.
\end{desiderata}

\subsubsection{Inter-constraint coordination}
It would be nice if the hyperlens associated with
\begin{align*}
    x &= a+b \\
    y &= x+c
\end{align*}
behaved ``similarly'' to the hyperlens associated with
\begin{align*}
    x &= b+c \\
    y &= a+x
\end{align*}
in the sense that an update to $y$ in either system resulted in the same
updates to $a$, $b$, and $c$. This is nice from a language design point of
view because it means you need not introduce separate $+$ functions for each
arity, and is nice from a usability point of view because it means that
there is no price to pay for modularity: you can split up your code into
whatever units make sense to you and get the same program out.

\begin{desiderata}
    Allow constraints to interact during system update.
\end{desiderata}
