\documentclass{article}
\usepackage{amsmath}
\usepackage{tikz}
\usepackage{tikz-qtree}
\newcommand{\lens}{\leftrightarrow}
\newcommand{\lget}{\mathsf{get}}
\newcommand{\lput}{\mathsf{put}}
\begin{document}
{\noindent\large Daniel Wagner}\\[3ex]

The problem of synchronization is ubiquitous in computing applications. In
the database community, the so-called view-update problem involves keeping a
database and the result of a query to that database in sync as updates occur
to the database and to the query's result. In large software systems, two
applications often want to share data, but have disparate preferences for
file format; or, various applications, all attempting to solve the same
problem, may want to share user data. (For example, browsers like Internet
Explorer, Firefox, and Chrome often have facilities for importing bookmarks
from their competitors.) Users with multiple computers may wish to
synchronize directory hierarchies among their computers. Large websites
often offer programming interfaces that can be modeled as presenting a
small, modifiable view of their large set of data. Even a single program
running on a single computer often keep various in-memory data structures
storing related data, and maintain synchronization invariants between the
structures. I work towards the goal of making the maintenance of
synchronized data as painless as possible, so that ephemeral data structures
and persistent data alike can benefit from formal correspondence guarantees.

There is a fair amount of preexisting work on this problem. The database
community in particular has a significant body of work on the problem of
synchronizing relational data. The framework of asymmetric lenses aims to
extend these ideas to other familiar data structures often available in
programming languages (with varying but often fairly extensive language
integration), such as strings and algebraic data types.
% TODO: citations, more expansive related work

Broadly speaking, there are three avenues of inquiry involved in designing a
good bidirectional language (that are of course often quite intertwined):
\begin{enumerate}
    \item Designing a model. Here the goal is to carefully state, in
        abstract terms, the components a bidirectional program is expected
        to provide, along with a statement of the behavioral guarantees you
        expect from these components.

        The classical choice here is to discuss a lens between two sets $A$
        and $B$ as having two components:
        \begin{align*}
            \lget &\in A \to B \\
            \lput &\in B \times A \to A
        \end{align*}
        That is, we view $A$ as a database and $B$ as some summary of the
        database; we demand that a lens provide a function to compute the
        summary and a function to take a modified summary and produce a
        modification to the database.

        The demand, then, is that these functions really synchronize the
        database and the summary. That is, if we provide the same summary as
        we used to have, nothing should happen to the database, and if we
        provide the database we just got from an update, our summaries
        should match:
        \begin{align*}
            \lget(\lput(b,a)) &= b \\
            \lput(\lget(a),a) &= a
        \end{align*}

        One thrust of my research is to challenge this classical approach,
        varying the behavioral demands or the components required from a
        bidirectional program, and see what happens.
    \item Exploring algebraic properties. Once a model is fixed, one can
        ask algebraic questions. In almost all cases, the class of models
        forms a category, and as a result it is possible to use intuitions
        from category theory to ask good questions about what operations are
        implementable.

        One key property required to take advantage of category theoretical
        insights is that the models be compositional: one must show that, in
        whatever model has been chosen, the process of running first one and
        then another lens in sequence provides the same interface and
        satisfies the same behavioral guarantees when taken as a whole as
        each of the pieces do.  (Having this operation available is also a
        quite practical programming advantage: it is common to create a lens
        for a large and complicated transformation by combining many small
        lenses doing easy-to-understand portions of the transformation.)

        For example, there is a fairly standard categorical understanding of
        various data structures constructed from sums, products, recursion,
        and similar constructs. Exploring which of the operations required
        to support these features are available as bidirectional programs
        says quite a lot about the power of the model chosen.
    \item Inventing syntax. This is all about exploring which programs can
        be bidirectionalized, and inventing a library of independent,
        understandable lenses that can be combined to write them. The
        algebraic study often gives a head start here, but there are many
        practical functions that people wish to compute that are not
        algebraically-inspired. When done well, this design phase will
        validate a good choice of model: a model is only useful and helpful
        if it tells you something about the programs you want to write.

        A big part of this phase involves the problem of alignment: how do
        we detect the relationship between old versions of the data and new
        versions, and once we have detected this, how do we use the
        information? Much of the answer here lies in the design and use of
        good heuristics and in providing programmer control over their use.
\end{enumerate}

Thus, my interest is in the intersection of bidirectional program design and
language design: my goal is to create useful, usable languages with strong
behavioral guarantees that can be used to synchronize multiple data sources.

\end{document}
