\documentclass{article}
\usepackage{amsmath}
\begin{document}
{\noindent Daniel Wagner}\\[0ex]

Recent years have seen increased interest in the area of bidirectional
programming. Broadly speaking, the problem domain involves maintaining a
connection between two different representations of otherwise very similar
information. The strong connections between an in-memory representation of a
data structure and its serialized form; a piece of source code and its
parsed abstract syntax tree; tool-specific configuration formats and a
common configuration format; a database and some particular summary of
interest; or two distant but partially replicated computers are all examples
of areas where two pieces of data are very similar. In each case, one would
like the two pieces of data to stay ``in synch'': modifications to one piece
should be propagated and reflected in the other.

At the moment, one common way of tackling this problem is to design by hand
two programs that work together. Calling the two pieces of data $X$ and $Y$,
the first program translates updates to $X$ into updates to $Y$, and the
second translates in the other direction, turning updates to $Y$ into
updates to $X$. (Taking the example of connecting a piece of source code and
its abstract syntax tree from above, these two programs might be a parser
and a pretty-printer.) However, this approach is messy and error-prone, and
it can be difficult to extract behavioral guarantees that the two programs
work together in appropriate ways. The goal of bidirectional programming
language design, then, is to create a language in which each term can be
interpreted in two ways and which makes it easy to prove that the two
different interpretations work together appropriately.

One well-studied approach to bidirectional programming is the framework of
\emph{asymmetric}, \emph{state-based} lenses. In this model, one of the two
repositories is a \emph{source}, and the other is a \emph{view} of or
\emph{query} on the other: that is, it can be completely reconstructed from
the other without additional outside information. Updates to the source can
be easily propagated to the view by rerunning the query, but propagating
updates from the view to the source can require significant sophistication.
The basic theory of lenses offers significant insight into the construction
of correct, convenient tools for this backward propagation; however, there
are several practical considerations which encourage further research. My
efforts are focused on four of these: size, symmetry, syntax, and alignment.

{\bf Size} When the two repositories being connected reside on separate
computers, transferring even a small copy of a repository across the network
to the other computer can take quite a noticeable amount of time. Mitigating
this effect typically involves using a compression scheme; empirically, one
of the most effective ones involves noting ``what has changed'' since
previous incarnations of the synchronization tool. Most file system
synchronizers and revision control systems have some such notion internally.
It seems likely that a practical tool based on lenses will need to use
similar tricks; so, for the theory to faithfully model such a tool, it must
model not just repository states but also repository edits and how they
affect the states.

{\bf Symmetry} The standard theory of lenses is asymmetric: one repository
is assumed to contain \emph{strictly more} information than the other. This
is the case for a surprising number of real-world bidirectional settings;
however, there certainly are some settings where each repository contains
information not available in the other, yet it is desirable to synchronize
the common parts. Designing a pair of lenses that synchronizes the two
repositories of interest with a third repository---freshly designed to
contain all the information available in either of the two existing
ones---is messy and error-prone for all the same reasons that designing a
pair of unidirectional programs is. One instead desires a symmetric theory
of lenses in which a single object can connect mutually incomplete
repositories.

{\bf Syntax} All lens theories begin with a foundation that includes an
interface that lenses must implement together with a set of behavioral laws
restricting which implementations are considered reasonable. Getting the
right definition for this foundation is critical, and many theories spend
significant effort developing and motivating the interface and behavioral
laws. A practical theory should also include instantiations of this
framework -- particular lenses and lens combinators that are proven to
satisfy the interface and behavioral laws. This collection of instantiations
forms the basis for bidirectional language syntax, and hence should include
tools for handling common data types and for sequencing.

{\bf Alignment} One of the most difficult problems in lens research involves
correctly associating data between versions of a repository. For example,
consider a function which merges changes to a lower-cased string back into
the mixed-case string. Many cases are easy to get right:
\begin{align*}
    \mathit{merge}(\mathtt{uppercasedword},\mathtt{UpperCasedQord})
        &= \mathtt{UpperCasedWord} \\
    \mathit{merge}(\mathtt{uppercased},\mathtt{UpperCasedWord})
        &= \mathtt{UpperCased} \\
    \mathit{merge}(\mathtt{uppercasedsentence},\mathtt{UpperCasedWord})
        &= \mathtt{UpperCasedSentence}
\end{align*}
If done naively---e.g. upper-casing letters in the new string in the same
places as there were upper-case letters in the old string---it is easy to
get a surprising result like $\mathit{merge}(\mathtt{uppercasedword},
\mathtt{UperCasedWord}) = \mathtt{uppeRcaseDword}$.  The problem is already
tricky with lists (though for lack of space we stop with this simple
example), and gets even trickier with more complicated data structures.

My research is focused on making lens languages practical, beginning with
the problems described above: reducing sizable communication costs,
providing symmetric synchronization tools to connect mutually incomplete
repositories, ensuring that the foundations can be instantiated with a
usable syntax, and developing techniques for identifying and using
information about the alignment between old and new versions of data.
\end{document}
