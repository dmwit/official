\documentclass{article}

\usepackage{xspace}
\newcommand{\undefined}{\ensuremath{\bot}\xspace}
\newcommand{\apply}{\ensuremath{\mathbf{apply}}\xspace}
\newcommand{\internal}{\ensuremath{\mathbf{internal}}\ }
\newcommand{\leaf}{\ensuremath{\mathbf{leaf}}\ }
\newcommand{\contains}{\ \ensuremath{\mathbf{contains}}\ }
\newcommand{\relR}{\mathrel{R}}

\begin{document}
% TODO: title
\title{Changes}
\author{Daniel Wagner}
\date{January 21, 2011}
\maketitle
\begin{abstract}
    % TODO
    We discuss various theories of changes, identifying various goals of
    these theories and evaluating tradeoffs between different
    representations of changes.
\end{abstract}

\section{Introduction}
\begin{itemize}
    \item lots of reasons to muck about with edits
        \begin{itemize}
            \item revision control
            \item text editors (undo, redo)
            \item \ldots also, other kinds of editors (worth mentioning?)
            \item synchronization
            \item database updates
        \end{itemize}
    \item each has its own success criteria
        \begin{itemize}
            \item ease of reasoning
            \item conflict detection
            \item merging updates
            \item reordering/concurrent application
            \item computing changes from two given objects
            \item human readability
            \item human writability
            \item ease of generation
        \end{itemize}
    \item huge range of possible data objects and edit languages
        \begin{itemize}
            \item one end of the spectrum: simple sequences + diff/patch
                format
            \item another end: any data type + write a program in C that
                edits it
            \item in between: relations + SQL
            \item in this paper: labelled, unordered trees + a variety of
                languages that run across the spectrum from very simple to
                very expressive
        \end{itemize}
    \item focus on three uses:
        \begin{itemize}
            \item formalizing the internals of a revision control system
            \item efficiently computing diffs (for use as input to said RCS)
            \item identifying re-orderings (to be used when optimizing
                multiple updates)
        \end{itemize}
    \item misc. commonalities, typographical conventions, etc.
        \begin{itemize}
            \item $\apply(e,d)$ to apply edit $e$ (in the current edit
                language) to data $d$
            \item \undefined for undefined; always a possible result of
                $\apply$
            \item assume set of labels $L$, metavariable $\ell$ for
                individual labels
            \item metavariable $p$ for paths
        \end{itemize}
\end{itemize}

\section{Semantics and conflict detection}
\begin{itemize}
    \item want to define
        \begin{itemize}
            \item repository (or patch history)
            \item patch
            \item working copy
            \item successful merge, conflict (when synchronizing two
                repositories)
        \end{itemize}
    \item as usual, for trees, but will generalize the approach somewhat -
        definitions can be applied to repositories tracking other structures
    \item somewhat unusual representation of trees
        \begin{itemize}
            \item collection of assertions + well-formedness invariant
            \item assertions have form $\internal p$ or $\leaf p \contains
                C$, where $C$ drawn from some set of possible contents
            \item paths are just $\ell$ or $\ell/p$
            \item well-formedness invariant says for each $p$:
                \begin{itemize}
                    \item if $\leaf p \contains C$, then no $C' \ne C$ such
                        that $\leaf p \contains C'$
                    \item if $\leaf p \contains C$, then no $\internal p$
                    \item if $\internal p$, then no $\leaf p \contains C$
                    \item if $\leaf p/\ell \contains C$ or $\internal
                        p/\ell$, then $\internal p$
                    \item essentially just enforcing that it's
                        tree-structured
                \end{itemize}
        \end{itemize}
    \item patches should preserve the invariants
        \begin{itemize}
            \item can achieve this with a precondition stating assertions
                that must already be there and assertions that must not be
                there
            \item example: to add leaf node, must have parent internal node,
                and must not have any pre-existing leaf node
            \item patch is therefore a triple of assertion sets $(S,E,T)$
                \begin{itemize}
                    \item all assertions in $S$ must already exist
                    \item no assertions in $E$ must exist unless they're
                        also in $S$
                    \item effect of the patch is to remove $S$ and add $T$
                    \item maybe $S \cap T \ne \emptyset$: the parent
                        directory in our example
                \end{itemize}
            \item $\apply((S,E,T),d)=d \triangle S \triangle T$, except when
                it's \undefined
        \end{itemize}
    \item repository: multiset of patches
    \item aside: WHY use this weird representation?
        \begin{itemize}
            \item uniform mathematical formalism for a variety of repository
                structures
            \item to get different structures, modify the assertion language
                and the invariant
            \item in a later subsection, show this flexibility by changing
                the tree content or adding metadata
            \item even with those changes, the definitions of the later bits
                don't change -- they're independent of the actual data
        \end{itemize}
    \item comments on repositories
        \begin{itemize}
            \item build up the working copy from empty set of assertions
            \item not all repositories make sense
            \item \emph{consistent} when there is an order in which the
                patches can be applied
            \item any order ends at the same place (commutativity of
                $\triangle$)
        \end{itemize}
    \item the payload: syncing two repositories
        \begin{itemize}
            \item sets $R_1, R_2$ of patches
            \item set $P \subset R_2 \setminus R_1$ of patches to pull from
                $R_2$ to $R_1$
            \item define: \emph{conflict} when $R_1 \cup P$ is not
                consistent
            \item example goes here, plus how to deal with it (expanding $P$
                or human intervention)
        \end{itemize}
    \item discussion
        \begin{itemize}
            \item theory as given so far impractical: infinite sets
                \begin{itemize}
                    \item in practice: representations of a few common kinds
                        of patches
                    \item appeal to the model when writing \apply or the
                        function that checks preconditions
                    \item example of a patch representation vs. patch triple
                    \item for consistency, let this example be the edits
                        from the next paper (sans inverses)
                    \item compound patches
                    \item similarly, more concrete repository, appealing to
                        theory when verifying algorithm correctness
                \end{itemize}
            \item extensions: maybe do line-based files as example of this
            \item other impracticality: algorithms for change detection,
                consistency checking, minimal pulls not obviously
                forthcoming (but see next section for one of these)
        \end{itemize}
\end{itemize}

\section{Change detection}
\begin{itemize}
    \item have two working copies
        \begin{itemize}
            \item one from repository
            \item one on disk
        \end{itemize}
    \item task: generate patch to convert one to the other
        \begin{itemize}
            \item using patches like those defined above
            \item compound patches, inverse patches
            \item footnote explaining that $(S,E,T)^{-1}=(T,E,S)$
            \item problem actually ill-defined for now
            \item obvious patch: delete whole tree, insert whole new tree
            \item introduce cost function, find minimal patch
        \end{itemize}
    \item lots of approaches
        \begin{itemize}
            \item using variants of these patches
            \item assuming node identity or not
            \item various cost functions
            \item figure out some other distinctions
            \item a primary distinction: allowing moves/copies or not
                \begin{itemize}
                    \item (actually, real question is how cost function
                        behaves on moves/copies compared to
                        insertions/deletions)
                    \item disallow them: surprise the humans
                    \item allow them: NP-hard
                \end{itemize}
            \item cite other approaches
        \end{itemize}
    \item some decisions:
        \begin{itemize}
            \item exactly the patches defined above (+ inverses)
            \item no node identities
            \item cost function: constant for each kind of patch, plus
                distance function for replacements
            \item on allowing moves/copies: heuristic (potentially
                non-optimal); will discuss approximation points as they
                arise
            \item will operate on $d_1$, $d_2$ as if they were trees, not
                sets of assertions
        \end{itemize}
    \item main idea
        \begin{itemize}
            \item find minimal alignment
            \item convert alignment to minimal edit
        \end{itemize}
    \item alignment
        \begin{itemize}
            \item operator $(-)^+$ adds distinct internal node with no
                parent to a tree
            \item an \emph{alignment} is a minimal edge cover of the
                complete (unweighted) bipartite graph whose parts are the
                nodes of $d_1^+$ and $d_2^+$
            \item edge $(a,b)$ signals that node $a$ in $d_1$ became node
                $b$ in $d_2$ by the edit corresponding to this alignment
            \item special cases: edge $(a,+)$ means $a$ was deleted, $(+,b)$
                means $b$ was created new, $(+,+)$ means nothing
        \end{itemize}
    \item cost of alignment = cost of resulting patch
    \item minimal alignment? gotta know how to convert to a patch
        \begin{itemize}
            \item writing atomic update + ordering updates
            \item writing atomic update is by case analysis
                \begin{itemize}
                    \item $(a,+)$: delete
                    \item $(+,b)$: insert
                    \item $(a,b)$: move/copy/inverse copy, maybe followed by
                        update\ldots unless their parents have an edge
                    \item some logic that evaluates such situations to
                        minimize number of moves/copies
                \end{itemize}
            \item ordering updates: constraints generated during case
                analysis above, then toposorted
            \item give example where ordering matters (and helps)
            \item comment: can't reach all edits this way
                \begin{itemize}
                    \item TODO: properties of edits we can reach
                    \item description of edits we can't reach, plus how it
                        might help make a smaller cost
                    \item one point of approximation
                \end{itemize}
        \end{itemize}
    \item observation: no such thing as ``cost of an edge'' in the
        alignment; the cost of the updates generated by that edge depends on
        what other edges are available
    \item searching all possible minimal covers: exponential
    \item approach: upper and lower bounds on edge costs
    \item for now, skip how these are computed -- clever, but not central to
        the idea
    \item prune obviously bad edges, update cost bounds
        \begin{itemize}
            \item TODO: list pruning rules here
        \end{itemize}
    \item weighted matching on resulting (now, weighted) bipartite graph
        \begin{itemize}
            \item eh? resulting weighted graph? each edge has two weights
            \item use lower bounds as estimate of actual cost of each edge
                (or upper bounds, or average, or whatever)
            \item another point of approximation
        \end{itemize}
    \item discussion
        \begin{itemize}
            \item bad things
                \begin{itemize}
                    \item both complicated and heuristic (no theoretical
                        guarantees, so potentially unpredictable behavior)
                    \item somewhat inflexible to edit variants -- conversion
                        to (minimal) edit and cost estimation quite
                        dependent on this exact set existing
                \end{itemize}
            \item good things
                \begin{itemize}
                    \item fairly fast
                    \item extensible to files with tree-structured data
                        (i.e. most non-binary files)
                    \item flexible cost model
                    \item move and copy big win not only for revision
                        control, but for (stateful) lenses, too
                \end{itemize}
            \item somewhat philosophical whether the edits we can't reach
                from an alignment are desirable or not
        \end{itemize}
\end{itemize}

\section{Reordering}

\section{Conclusion}

\end{document}
